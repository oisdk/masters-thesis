\documentclass[acmsmall,screen]{acmart}

\definecolor{AHKeywordColour}    {cmyk}{0.86,0.91,0,0.04}
\definecolor{AHFunctionColour}   {cmyk}{0.64,0,0.95,0.40}
\definecolor{AHConstructorColour}{cmyk}{0,0.83,1,0.70}
\definecolor{AHNumberColour}     {HTML}{A020F0}
\definecolor{AHStringColour}     {HTML}{B22222}
\definecolor{TodoColour}         {rgb} {0.9,0.9,0.98}

\PassOptionsToPackage{full}{textcomp}

% Tikz
\usepackage{tikz}
\usepackage{forest}
\usetikzlibrary{decorations.pathreplacing}
\usetikzlibrary{calc}
\usetikzlibrary{cd}
\usepackage{pgfplots}
\usepackage{pgfplotstable}
\pgfplotstableset{col sep=comma}
\pgfplotsset{compat=1.3}

\usepackage{wrapfig}
\usepackage{cancel}

\usepackage{lineno}

\usepackage{cleveref}

% For some of the operators I define myself I use this package.
\usepackage{pict2e}

% This package lets us import code snippets into the document.
\usepackage{catchfilebetweentags}

% Packages with maths symbols.
\usepackage{mathtools}
\usepackage{turnstile}
\usepackage{stmaryrd}
\usepackage{latexsym}

\newtagform{nowidth}{\llap\bgroup(}{)\egroup}

% Allow for blackboard bold 0 and 1.
\DeclareMathAlphabet{\numberbb}{U}{bbold}{m}{n}

% Theorem environments
\theoremstyle{acmdefinition}
\newtheorem{definition}{Definition}


\newtheorem*{remark}{Remark}

% Maths fonts
\usepackage{mathrsfs}
\usepackage{bbm}
\let\Bbbk\relax % Need to avoid defining \Bbbk here, because it's defined in the
                % acmart template.

% Needed to typeset Agda's stuff (I think; I haven't actually tried removing it recently).
\usepackage[english]{babel}

% Handy package for on-the-fly resizing.
\usepackage{relsize}

% Does links and stuff.
\usepackage{hyperref}

\usepackage{cleveref}

% Citations
\usepackage{natbib}
\citestyle{acmauthoryear}

% Multiple figures
\usepackage{subcaption}
\usepackage{dblfloatfix}
\usepackage{multicol}
\setlength{\multicolsep}{6.0pt plus 2.0pt minus 1.5pt}% 50% of original values


\newcommand{\ExecuteMetaDataInline}[2][\jobname]{\ExecuteMetaData[#1]{#2}\ignorespaces}

\newcommand{\sameline}[1]{\sloppy{\mbox{#1}}}
\newcommand{\mathsameline}[1]{\sloppy{\({#1}\)}}

\newcommand{\twocolcode}[2]{\noindent%
\begin{minipage}[t][][b]{.49\linewidth}\vspace{-0.7\baselineskip}%
    #1%
  \end{minipage}%
  \begin{minipage}[t][][b]{.49\linewidth}\vspace{-0.7\baselineskip}%
    #2%
  \end{minipage}\\\noindent%
}

% Todo notes
\usepackage[backgroundcolor=TodoColour,textwidth=1.15cm]{todonotes}
\setlength{\marginparwidth}{1.22cm}
\newcommand{\mtodo}[1]{\todo[noinline,noline,size=tiny]{\tiny #1}}

% Arrow for hyperfunctions. Can't decide on best one.

% loop shape:
\newcommand{\hyparrow}{\looparrowright}
\newcommand{\hypmarrow}[1]{%
  \mathrel{%
    \makebox[0pt][c]{%
      \raisebox{-0.35em}{\(\;\;\scriptstyle#1\)}%
    }%
    \raisebox{0.15em}{\(\looparrowright\)}%
  }%
}
\tikzcdset{
  arrow style=tikz,
  diagrams={>={Straight Barb[scale=0.8]}},
}
\usepackage{amsmath}
\usepackage{amsthm}
\renewcommand{\qed}{\ensuremath{\hfill\blacksquare}}
\makeatletter
\newcommand{\eqnum}{\leavevmode\textup{\tagform@{\theequation}}}
\makeatother

\usepackage{algorithm2e}
\newenvironment{agdalisting}
{

\refstepcounter{equation}\noindent\raisebox{0pt}[0pt][0pt]{%
\llap{%
\parbox[t][][t]{.7\marginparwidth}{%
\raggedleft%
{\eqnum}}\hspace{\marginparsep}}}%
\begin{minipage}{\linewidth}\centering\vspace{.7\abovedisplayskip}
}
{\end{minipage}\vspace{\belowdisplayshortskip}}
\newenvironment{agdalisting*}
{\noindent\begin{minipage}{\linewidth}\centering\vspace{.7\abovedisplayskip}}{\end{minipage}\vspace{\belowdisplayshortskip}}

\usepackage{environ}
\NewEnviron{marginagdalisting}[1]{%
  \refstepcounter{equation}
  \marginpar{\centering%
    \BODY\vspace{-0.8\baselineskip}\\%
    Listing {\theequation}: #1%
  }%
}

% ----------------------------------------------------------------------
% Some useful commands when doing highlighting of Agda code in LaTeX.
% ----------------------------------------------------------------------

\ProvidesPackage{agda}

\RequirePackage{ifxetex, ifluatex, xifthen, xcolor, polytable, etoolbox,
                calc, environ}

% https://tex.stackexchange.com/questions/47576/combining-ifxetex-and-ifluatex-with-the-logical-or-operation
\newif\ifxetexorluatex
\ifxetex
  \xetexorluatextrue
\else
  \ifluatex
    \xetexorluatextrue
  \else
    \xetexorluatexfalse
  \fi
\fi

% ----------------------------------------------------------------------
% Options

\DeclareOption{bw}   {\newcommand{\AgdaColourScheme}{bw}}
\DeclareOption{conor}{\newcommand{\AgdaColourScheme}{conor}}

\newif\if@AgdaEnableReferences\@AgdaEnableReferencesfalse
\DeclareOption{references}{
  \@AgdaEnableReferencestrue
}

\newif\if@AgdaEnableLinks\@AgdaEnableLinksfalse
\DeclareOption{links}{
  \@AgdaEnableLinkstrue
}

% If the "nofontsetup" option is used, then the package does not
% select any fonts, and it does not change the font encoding.
\newif\if@AgdaSetupFonts\@AgdaSetupFontstrue
\DeclareOption{nofontsetup}{
  \@AgdaSetupFontsfalse
}

% If the "noinputencodingsetup" option is used, then the package does
% not change the input encoding, and it does not load the `ucs`
% package.
\newif\if@AgdaSetupInputEncoding\@AgdaSetupInputEncodingtrue
\DeclareOption{noinputencodingsetup}{
  \@AgdaSetupInputEncodingfalse
}

\ProcessOptions\relax

% ----------------------------------------------------------------------
% Input encoding setup

\if@AgdaSetupInputEncoding
  \ifxetexorluatex

    \providecommand{\DeclareUnicodeCharacter}[2]{\relax}

  \else

    \RequirePackage{ucs}
    \RequirePackage[utf8x]{inputenc}

  \fi
\fi

% ----------------------------------------------------------------------
% Font setup

\tracinglostchars=2 % If the font is missing some symbol, then say
                    % so in the compilation output.

\if@AgdaSetupFonts
  % XeLaTeX or LuaLaTeX
  \ifxetexorluatex

    % Hack to get the amsthm package working.
    % https://tex.stackexchange.com/questions/130491/xelatex-error-paragraph-ended-before-tempa-was-complete
    \let\AgdaOpenBracket\[\let\AgdaCloseBracket\]
    \RequirePackage{fontspec}
    \let\[\AgdaOpenBracket\let\]\AgdaCloseBracket
    \RequirePackage{unicode-math}

    \setmainfont
      [ Ligatures      = TeX
      , BoldItalicFont = xits-bolditalic.otf
      , BoldFont       = xits-bold.otf
      , ItalicFont     = xits-italic.otf
      ]
      {xits-regular.otf}

    \setmathfont{xits-math.otf}
    \setmonofont[Mapping=tex-text]{FreeMono.otf}

    % Make mathcal and mathscr appear as different.
    % https://tex.stackexchange.com/questions/120065/xetex-what-happened-to-mathcal-and-mathscr
    \setmathfont[range={\mathcal,\mathbfcal},StylisticSet=1]{xits-math.otf}
    \setmathfont[range=\mathscr]{xits-math.otf}

  % pdfLaTeX
  \else

    % https://tex.stackexchange.com/questions/1774/how-to-insert-pipe-symbol-in-latex
    \RequirePackage[T1]{fontenc}

    \RequirePackage{amsfonts, amssymb}
    \RequirePackage[safe]{tipa} % See page 12 of the tipa manual for what
                                % safe does.

  \fi
\fi

% ----------------------------------------------------------------------
% Colour schemes.

\providecommand{\AgdaColourScheme}{standard}

% ----------------------------------------------------------------------
% References to code (needs additional post-processing of tex files to
% work, see wiki for details).

\if@AgdaEnableReferences
  \RequirePackage{catchfilebetweentags, xstring}
  \newcommand{\AgdaRef}[2][]{%
    \StrSubstitute{#2}{\_}{AgdaUnderscore}[\tmp]%
    \ifthenelse{\isempty{#1}}%
               {\ExecuteMetaData{AgdaTag-\tmp}}%
               {\ExecuteMetaData{#1}{AgdaTag-\tmp}}
  }
\fi

\providecommand{\AgdaRef}[2][]{#2}

% ----------------------------------------------------------------------
% Links (only done if the option is passed and the user has loaded the
% hyperref package).

\if@AgdaEnableLinks
  \@ifpackageloaded{hyperref}{

    % List that holds added targets.
    \newcommand{\AgdaList}[0]{}

    \newtoggle{AgdaIsElem}
    \newcounter{AgdaIndex}
    \newcommand{\AgdaLookup}[3]{%
      \togglefalse{AgdaIsElem}%
      \setcounter{AgdaIndex}{0}%
      \renewcommand*{\do}[1]{%
      \ifstrequal{#1}{##1}%
        {\toggletrue{AgdaIsElem}\listbreak}%
        {\stepcounter{AgdaIndex}}}%
      \dolistloop{\AgdaList}%
      \iftoggle{AgdaIsElem}{#2}{#3}%
    }

    \newcommand*{\AgdaTargetHelper}[1]{%
      \AgdaLookup{#1}%
        {\PackageError{agda}{``#1'' used as target more than once}%
                            {Overloaded identifiers and links do not%
                             work well, consider using unique%
                             \MessageBreak identifiers instead.}%
        }%
        {\listadd{\AgdaList}{#1}%
         \hypertarget{Agda\theAgdaIndex}{}%
        }%
    }

    \newcommand{\AgdaTarget}[1]{\forcsvlist{\AgdaTargetHelper}{#1}}

    \newcommand{\AgdaLink}[1]{%
      \AgdaLookup{#1}%
        {\hyperlink{Agda\theAgdaIndex}{#1}}%
        {#1}%
    }
  }{\PackageError{agda}{Load the hyperref package before the agda package}{}}
\fi

\providecommand{\AgdaTarget}[1]{}
\providecommand{\AgdaLink}[1]{#1}

% ----------------------------------------------------------------------
% Font styles.

\ifxetexorluatex
  \newcommand{\AgdaFontStyle}[1]{\ensuremath{\mathsf{#1}}}
  \ifthenelse{\equal{\AgdaColourScheme}{bw}}{
      \newcommand{\AgdaKeywordFontStyle}[1]{\underline{#1}}
  }{
      \newcommand{\AgdaKeywordFontStyle}[1]{\ensuremath{\mathsf{#1}}}
  }
  \newcommand{\AgdaStringFontStyle}[1]{\ensuremath{\texttt{#1}}}
  \newcommand{\AgdaCommentFontStyle}[1]{\ensuremath{\texttt{#1}}}
  \newcommand{\AgdaBoundFontStyle}[1]{\ensuremath{\mathit{#1}}}

\else
  \newcommand{\AgdaFontStyle}[1]{\textsf{#1}}
  \ifthenelse{\equal{\AgdaColourScheme}{bw}}{
      \newcommand{\AgdaKeywordFontStyle}[1]{\underline{#1}}
  }{
      \newcommand{\AgdaKeywordFontStyle}[1]{\textsf{#1}}
  }
  \newcommand{\AgdaStringFontStyle}[1]{\texttt{#1}}
  \newcommand{\AgdaCommentFontStyle}[1]{\texttt{#1}}
  \newcommand{\AgdaBoundFontStyle}[1]{\textit{#1}}
\fi

% ----------------------------------------------------------------------
% Colours.

% ----------------------------------
% The black and white colour scheme.
\ifthenelse{\equal{\AgdaColourScheme}{bw}}{

    % Aspect colours.
    \definecolor{AgdaComment}      {HTML}{000000}
    \definecolor{AgdaOption}       {HTML}{000000}
    \definecolor{AgdaKeyword}      {HTML}{000000}
    \definecolor{AgdaString}       {HTML}{000000}
    \definecolor{AgdaNumber}       {HTML}{000000}
    \definecolor{AgdaSymbol}       {HTML}{000000}
    \definecolor{AgdaPrimitiveType}{HTML}{000000}

    % NameKind colours.
    \definecolor{AgdaBound}                 {HTML}{000000}
    \definecolor{AgdaGeneralizable}         {HTML}{000000}
    \definecolor{AgdaInductiveConstructor}  {HTML}{000000}
    \definecolor{AgdaCoinductiveConstructor}{HTML}{000000}
    \definecolor{AgdaDatatype}              {HTML}{000000}
    \definecolor{AgdaField}                 {HTML}{000000}
    \definecolor{AgdaFunction}              {HTML}{000000}
    \definecolor{AgdaMacro}                 {HTML}{000000}
    \definecolor{AgdaModule}                {HTML}{000000}
    \definecolor{AgdaPostulate}             {HTML}{000000}
    \definecolor{AgdaPrimitive}             {HTML}{000000}
    \definecolor{AgdaRecord}                {HTML}{000000}
    \definecolor{AgdaArgument}              {HTML}{000000}

    % Other aspect colours.
    \definecolor{AgdaDottedPattern}     {HTML}{000000}
    \definecolor{AgdaUnsolvedMeta}      {HTML}{D3D3D3}
    \definecolor{AgdaTerminationProblem}{HTML}{BEBEBE}
    \definecolor{AgdaIncompletePattern} {HTML}{D3D3D3}
    \definecolor{AgdaError}             {HTML}{696969}

    % Misc.
    \definecolor{AgdaHole}              {HTML}{BEBEBE}

% ----------------------------------
% Conor McBride's colour scheme.
}{ \ifthenelse{\equal{\AgdaColourScheme}{conor}}{

    % Aspect colours.
    \definecolor{AgdaComment}      {HTML}{B22222}
    \definecolor{AgdaOption}       {HTML}{000000}
    \definecolor{AgdaKeyword}      {HTML}{000000}
    \definecolor{AgdaString}       {HTML}{000000}
    \definecolor{AgdaNumber}       {HTML}{000000}
    \definecolor{AgdaSymbol}       {HTML}{000000}
    \definecolor{AgdaPrimitiveType}{HTML}{0000CD}

    % NameKind colours.
    \definecolor{AgdaBound}                 {HTML}{A020F0}
    \definecolor{AgdaGeneralizable}         {HTML}{A020F0}
    \definecolor{AgdaInductiveConstructor}  {HTML}{8B0000}
    \definecolor{AgdaCoinductiveConstructor}{HTML}{8B0000}
    \definecolor{AgdaDatatype}              {HTML}{0000CD}
    \definecolor{AgdaField}                 {HTML}{8B0000}
    \definecolor{AgdaFunction}              {HTML}{006400}
    \definecolor{AgdaMacro}                 {HTML}{006400}
    \definecolor{AgdaModule}                {HTML}{006400}
    \definecolor{AgdaPostulate}             {HTML}{006400}
    \definecolor{AgdaPrimitive}             {HTML}{006400}
    \definecolor{AgdaRecord}                {HTML}{0000CD}
    \definecolor{AgdaArgument}              {HTML}{404040}

    % Other aspect colours.
    \definecolor{AgdaDottedPattern}     {HTML}{000000}
    \definecolor{AgdaUnsolvedMeta}      {HTML}{FFD700}
    \definecolor{AgdaTerminationProblem}{HTML}{FF0000}
    \definecolor{AgdaIncompletePattern} {HTML}{A020F0}
    \definecolor{AgdaError}             {HTML}{F4A460}

    % Misc.
    \definecolor{AgdaHole}              {HTML}{9DFF9D}

% ----------------------------------
% The standard colour scheme.
}{
    % Aspect colours.
    \definecolor{AgdaComment}      {HTML}{B22222}
    \definecolor{AgdaOption}       {HTML}{000000}
    \definecolor{AgdaKeyword}      {HTML}{CD6600}
    \definecolor{AgdaString}       {HTML}{B22222}
    \definecolor{AgdaNumber}       {HTML}{A020F0}
    \definecolor{AgdaSymbol}       {HTML}{404040}
    \definecolor{AgdaPrimitiveType}{HTML}{0000CD}

    % NameKind colours.
    \definecolor{AgdaBound}                 {HTML}{000000}
    \definecolor{AgdaGeneralizable}         {HTML}{000000}
    \definecolor{AgdaInductiveConstructor}  {HTML}{008B00}
    \definecolor{AgdaCoinductiveConstructor}{HTML}{8B7500}
    \definecolor{AgdaDatatype}              {HTML}{0000CD}
    \definecolor{AgdaField}                 {HTML}{EE1289}
    \definecolor{AgdaFunction}              {HTML}{0000CD}
    \definecolor{AgdaMacro}                 {HTML}{458B74}
    \definecolor{AgdaModule}                {HTML}{A020F0}
    \definecolor{AgdaPostulate}             {HTML}{0000CD}
    \definecolor{AgdaPrimitive}             {HTML}{0000CD}
    \definecolor{AgdaRecord}                {HTML}{0000CD}
    \definecolor{AgdaArgument}              {HTML}{404040}

    % Other aspect colours.
    \definecolor{AgdaDottedPattern}     {HTML}{000000}
    \definecolor{AgdaUnsolvedMeta}      {HTML}{FFFF00}
    \definecolor{AgdaTerminationProblem}{HTML}{FFA07A}
    \definecolor{AgdaIncompletePattern} {HTML}{F5DEB3}
    \definecolor{AgdaError}             {HTML}{FF0000}

    % Misc.
    \definecolor{AgdaHole}              {HTML}{9DFF9D}
}}

% ----------------------------------------------------------------------
% Commands.

\newcommand{\AgdaNoSpaceMath}[1]
    {\begingroup\thickmuskip=0mu\medmuskip=0mu#1\endgroup}

% Aspect commands.
\newcommand{\AgdaComment}     [1]
    {\AgdaNoSpaceMath{\AgdaCommentFontStyle{\textcolor{AgdaComment}{#1}}}}
\newcommand{\AgdaOption}      [1]
    {\AgdaNoSpaceMath{\AgdaCommentFontStyle{\textcolor{AgdaOption}{#1}}}}
\newcommand{\AgdaKeyword}     [1]
    {\AgdaNoSpaceMath{\AgdaKeywordFontStyle{\textcolor{AgdaKeyword}{#1}}}}
\newcommand{\AgdaString}      [1]
    {\AgdaNoSpaceMath{\AgdaStringFontStyle{\textcolor{AgdaString}{#1}}}}
\newcommand{\AgdaNumber}      [1]
    {\AgdaNoSpaceMath{\AgdaFontStyle{\textcolor{AgdaNumber}{#1}}}}
\newcommand{\AgdaSymbol}      [1]
    {\AgdaNoSpaceMath{\AgdaFontStyle{\textcolor{AgdaSymbol}{#1}}}}
\newcommand{\AgdaPrimitiveType}[1]
    {\AgdaNoSpaceMath{\AgdaFontStyle{\textcolor{AgdaPrimitiveType}{#1}}}}
% Note that, in code generated by the LaTeX backend, \AgdaOperator is
% always applied to a NameKind command.
\newcommand{\AgdaOperator}    [1]{#1}

% NameKind commands.

% The user can control the typesetting of (certain) individual tokens
% by redefining the following command. The first argument is the token
% and the second argument the thing to be typeset (sometimes just the
% token, sometimes \AgdaLink{<the token>}). Example:
%
%   \usepackage{ifthen}
%
%   % Insert extra space before some tokens.
%   \DeclareRobustCommand{\AgdaFormat}[2]{%
%     \ifthenelse{
%       \equal{#1}{≡⟨} \OR
%       \equal{#1}{≡⟨⟩} \OR
%       \equal{#1}{∎}
%     }{\ }{}#2}
%
% Note the use of \DeclareRobustCommand.

\newcommand{\AgdaFormat}[2]{#2}

\newcommand{\AgdaBound}[1]
    {\AgdaNoSpaceMath{\AgdaBoundFontStyle{\textcolor{AgdaBound}{\AgdaFormat{#1}{#1}}}}}
\newcommand{\AgdaGeneralizable}[1]
    {\AgdaNoSpaceMath{\AgdaBoundFontStyle{\textcolor{AgdaGeneralizable}{\AgdaFormat{#1}{#1}}}}}
\newcommand{\AgdaInductiveConstructor}[1]
    {\AgdaNoSpaceMath{\AgdaFontStyle{\textcolor{AgdaInductiveConstructor}{\AgdaFormat{#1}{\AgdaLink{#1}}}}}}
\newcommand{\AgdaCoinductiveConstructor}[1]
    {\AgdaNoSpaceMath{\AgdaFontStyle{\textcolor{AgdaCoinductiveConstructor}{\AgdaFormat{#1}{\AgdaLink{#1}}}}}}
\newcommand{\AgdaDatatype}[1]
    {\AgdaNoSpaceMath{\AgdaFontStyle{\textcolor{AgdaDatatype}{\AgdaFormat{#1}{\AgdaLink{#1}}}}}}
\newcommand{\AgdaField}[1]
    {\AgdaNoSpaceMath{\AgdaFontStyle{\textcolor{AgdaField}{\AgdaFormat{#1}{\AgdaLink{#1}}}}}}
\newcommand{\AgdaFunction}[1]
    {\AgdaNoSpaceMath{\AgdaFontStyle{\textcolor{AgdaFunction}{\AgdaFormat{#1}{\AgdaLink{#1}}}}}}
\newcommand{\AgdaMacro}[1]
    {\AgdaNoSpaceMath{\AgdaFontStyle{\textcolor{AgdaMacro}{\AgdaFormat{#1}{\AgdaLink{#1}}}}}}
\newcommand{\AgdaModule}[1]
    {\AgdaNoSpaceMath{\AgdaFontStyle{\textcolor{AgdaModule}{\AgdaFormat{#1}{\AgdaLink{#1}}}}}}
\newcommand{\AgdaPostulate}[1]
    {\AgdaNoSpaceMath{\AgdaFontStyle{\textcolor{AgdaPostulate}{\AgdaFormat{#1}{\AgdaLink{#1}}}}}}
\newcommand{\AgdaPrimitive}[1]
    {\AgdaNoSpaceMath{\AgdaFontStyle{\textcolor{AgdaPrimitive}{\AgdaFormat{#1}{#1}}}}}
\newcommand{\AgdaRecord}[1]
    {\AgdaNoSpaceMath{\AgdaFontStyle{\textcolor{AgdaRecord}{\AgdaFormat{#1}{\AgdaLink{#1}}}}}}
\newcommand{\AgdaArgument}[1]
    {\AgdaNoSpaceMath{\AgdaBoundFontStyle{\textcolor{AgdaArgument}{\AgdaFormat{#1}{#1}}}}}

% Other aspect commands.
\newcommand{\AgdaFixityOp}          [1]{\AgdaNoSpaceMath{$#1$}}
\newcommand{\AgdaDottedPattern}     [1]{\textcolor{AgdaDottedPattern}{#1}}
\newcommand{\AgdaUnsolvedMeta}      [1]
    {\AgdaFontStyle{\colorbox{AgdaUnsolvedMeta}{#1}}}
\newcommand{\AgdaTerminationProblem}[1]
    {\AgdaFontStyle{\colorbox{AgdaTerminationProblem}{#1}}}
\newcommand{\AgdaIncompletePattern} [1]{\colorbox{AgdaIncompletePattern}{#1}}
\newcommand{\AgdaError}             [1]
    {\AgdaFontStyle{\textcolor{AgdaError}{\underline{#1}}}}
\newcommand{\AgdaCatchallClause}    [1]{#1} % feel free to change this

% Used to hide code from LaTeX.
%
% Note that this macro has been deprecated in favour of giving the
% hide argument to the code environment.
\long\def\AgdaHide#1{\ignorespaces}

% Misc.
\newcommand{\AgdaHole}[1]{\colorbox{AgdaHole}{#1}}

% ----------------------------------------------------------------------
% The code environment.

\newcommand{\AgdaCodeStyle}{}
% \newcommand{\AgdaCodeStyle}{\tiny}

\ifdefined\mathindent
  {}
\else
  \newdimen\mathindent\mathindent\leftmargini
\fi

% Adds the given amount of vertical space and starts a new line.
%
% The implementation comes from lhs2TeX's polycode.fmt, written by
% Andres Löh.
\newcommand{\Agda@NewlineWithVerticalSpace}[1]{%
  {\parskip=0pt\parindent=0pt\par\vskip #1\noindent}}

% Should there be space around code?
\newboolean{Agda@SpaceAroundCode}

% Use this command to avoid extra space around code blocks.
\newcommand{\AgdaNoSpaceAroundCode}{%
  \setboolean{Agda@SpaceAroundCode}{false}}

% Use this command to include extra space around code blocks.
\newcommand{\AgdaSpaceAroundCode}{%
  \setboolean{Agda@SpaceAroundCode}{true}}

% By default space is inserted around code blocks.
\AgdaSpaceAroundCode{}

% Sometimes one might want to break up a code block into multiple
% pieces, but keep code in different blocks aligned with respect to
% each other. Then one can use the AgdaAlign environment. Example
% usage:
%
%   \begin{AgdaAlign}
%   \begin{code}
%     code
%       code  (more code)
%   \end{code}
%   Explanation...
%   \begin{code}
%     aligned with "code"
%       code  (aligned with (more code))
%   \end{code}
%   \end{AgdaAlign}
%
% Note that AgdaAlign environments should not be nested.
%
% Sometimes one might also want to hide code in the middle of a code
% block. This can be accomplished in the following way:
%
%   \begin{AgdaAlign}
%   \begin{code}
%     visible
%   \end{code}
%   \begin{code}[hide]
%     hidden
%   \end{code}
%   \begin{code}
%     visible
%   \end{code}
%   \end{AgdaAlign}
%
% However, the result may be ugly: extra space is perhaps inserted
% around the code blocks.
%
% The AgdaSuppressSpace environment ensures that extra space is only
% inserted before the first code block, and after the last one (but
% not if \AgdaNoSpaceAroundCode{} is used). Example usage:
%
%   \begin{AgdaAlign}
%   \begin{code}
%     code
%       more code
%   \end{code}
%   Explanation...
%   \begin{AgdaSuppressSpace}
%   \begin{code}
%     aligned with "code"
%       aligned with "more code"
%   \end{code}
%   \begin{code}[hide]
%     hidden code
%   \end{code}
%   \begin{code}
%       also aligned with "more code"
%   \end{code}
%   \end{AgdaSuppressSpace}
%   \end{AgdaAlign}
%
% Note that AgdaSuppressSpace environments should not be nested.
%
% There is also a combined environment, AgdaMultiCode, that combines
% the effects of AgdaAlign and AgdaSuppressSpace.

% The number of the current/next code block (excluding hidden ones).
\newcounter{Agda@Current}
\setcounter{Agda@Current}{0}

% The number of the previous code block (excluding hidden ones), used
% locally in \Agda@SuppressEnd.
\newcounter{Agda@Previous}

% Is AgdaAlign active?
\newboolean{Agda@Align}
\setboolean{Agda@Align}{false}

% The number of the first code block (if any) in a given AgdaAlign
% environment.
\newcounter{Agda@AlignStart}

\newcommand{\Agda@AlignStart}{%
  \ifthenelse{\boolean{Agda@Align}}{%
    \PackageError{agda}{Nested AgdaAlign environments}{%
      AgdaAlign and AgdaMultiCode environments must not be
      nested.}}{%
    \setboolean{Agda@Align}{true}%
    \setcounter{Agda@AlignStart}{\value{Agda@Current}}}}

\newcommand{\Agda@AlignEnd}{\setboolean{Agda@Align}{false}}

\newenvironment{AgdaAlign}{%
  \Agda@AlignStart{}}{%
  \Agda@AlignEnd{}%
  \ignorespacesafterend}

% Is AgdaSuppressSpace active?
\newboolean{Agda@Suppress}
\setboolean{Agda@Suppress}{false}

% The number of the first code block (if any) in a given
% AgdaSuppressSpace environment.
\newcounter{Agda@SuppressStart}

% Does a "do not suppress space after" label exist for the current
% code block? (This boolean is used locally in the code environment's
% implementation.)
\newboolean{Agda@DoNotSuppressSpaceAfter}

\newcommand{\Agda@SuppressStart}{%
  \ifthenelse{\boolean{Agda@Suppress}}{%
    \PackageError{agda}{Nested AgdaSuppressSpace environments}{%
      AgdaSuppressSpace and AgdaMultiCode environments must not be
      nested.}}{%
    \setboolean{Agda@Suppress}{true}%
    \setcounter{Agda@SuppressStart}{\value{Agda@Current}}}}

% Marks the given code block as one that space should not be
% suppressed after (if AgdaSpaceAroundCode and AgdaSuppressSpace are
% both active).
\newcommand{\Agda@DoNotSuppressSpaceAfter}[1]{%
  % The use of labels is intended to ensure that LaTeX will provide a
  % warning if the document needs to be recompiled.
  \label{Agda@DoNotSuppressSpaceAfter@#1}}

\newcommand{\Agda@SuppressEnd}{%
  \ifthenelse{\value{Agda@SuppressStart} = \value{Agda@Current}}{}{%
    % Mark the previous code block in the .aux file.
    \setcounter{Agda@Previous}{\theAgda@Current-1}%
    \immediate\write\@auxout{%
      \noexpand\Agda@DoNotSuppressSpaceAfter{\theAgda@Previous}}}%
  \setboolean{Agda@Suppress}{false}}

\newenvironment{AgdaSuppressSpace}{%
  \Agda@SuppressStart{}}{%
  \Agda@SuppressEnd{}%
  \ignorespacesafterend}

\newenvironment{AgdaMultiCode}{%
  \Agda@AlignStart{}%
  \Agda@SuppressStart{}}{%
  \Agda@SuppressEnd{}%
  \Agda@AlignEnd{}%
  \ignorespacesafterend}

% Vertical space used for empty lines. By default \baselineskip.
\newlength{\AgdaEmptySkip}
\setlength{\AgdaEmptySkip}{\baselineskip}

% Extra space to be inserted for empty lines (the difference between
% \AgdaEmptySkip and \baselineskip). Used internally.
\newlength{\AgdaEmptyExtraSkip}

% The code environment.
%
% Code can be hidden by writing \begin{code}[hide].
%
% The implementation is based on plainhscode in lhs2TeX's
% polycode.fmt, written by Andres Löh.
\NewEnviron{code}[1][]{%
  % Conditionally hide the code.
  \ifthenelse{\equal{#1}{hide}}{}{%
    %
    % Conditionally emit space before the code block. Unconditionally
    % switch to a new line.
    \ifthenelse{\boolean{Agda@SpaceAroundCode} \and%
                \(\not \boolean{Agda@Suppress} \or%
                  \value{Agda@SuppressStart} = \value{Agda@Current}\)}{%
      \Agda@NewlineWithVerticalSpace{\abovedisplayskip}}{%
      \Agda@NewlineWithVerticalSpace{0pt}}%
    %
    % Indent the entire code block.
    % \advance\leftskip\mathindent%
    %
    % The code's style can be customised.
    \AgdaCodeStyle%
    %
    % Used to control the height of empty lines.
    \setlength{\AgdaEmptyExtraSkip}{\AgdaEmptySkip - \baselineskip}%
    %
    % The environment used to handle indentation (of individual lines)
    % and alignment.
    \begin{pboxed}%
      %
      % Conditionally preserve alignment between code blocks.
      \ifthenelse{\boolean{Agda@Align}}{%
        \ifthenelse{\value{Agda@AlignStart} = \value{Agda@Current}}{%
          \savecolumns}{%
          \restorecolumns}}{}%
      %
      % The code.
      \BODY%
    \end{pboxed}%
    %
    % Does the label Agda@DoNotSuppressAfter@<current code block
    % number> exist?
    \ifcsdef{r@Agda@DoNotSuppressSpaceAfter@\theAgda@Current}{%
      \setboolean{Agda@DoNotSuppressSpaceAfter}{true}}{%
      \setboolean{Agda@DoNotSuppressSpaceAfter}{false}}%
    %
    % Conditionally emit space after the code block. Unconditionally
    % switch to a new line.
    \ifthenelse{\boolean{Agda@SpaceAroundCode} \and%
                \(\not \boolean{Agda@Suppress} \or%
                  \boolean{Agda@DoNotSuppressSpaceAfter}\)}{%
      \Agda@NewlineWithVerticalSpace{\belowdisplayskip}}{%
      \Agda@NewlineWithVerticalSpace{0pt}}%
    %
    % Step the code block counter, but only for non-hidden code.
    \stepcounter{Agda@Current}}}

% Space inserted after tokens.
\newcommand{\AgdaSpace}{ }

% Space inserted to indent something.
\newcommand{\AgdaIndentSpace}{\AgdaSpace{}$\;\;$}

% Default column for polytable.
\defaultcolumn{@{}l@{\AgdaSpace{}}}

% \AgdaIndent expects a non-negative integer as its only argument.
% This integer should be the distance, in code blocks, to the thing
% relative to which the text is indented.
%
% The default implementation only indents if the thing that the text
% is indented relative to exists in the same code block or is wrapped
% in the same AgdaAlign or AgdaMultiCode environment.
\newcommand{\AgdaIndent}[1]{%
  \ifthenelse{#1 = 0
                \or
              \( \boolean{Agda@Align}
                   \and
                 \cnttest{\value{Agda@Current} - #1}{>=}{
                          \value{Agda@AlignStart}}
              \)}{\AgdaIndentSpace{}}{}}

% Underscores are typeset using \AgdaUnderscore{}.
\newcommand{\AgdaUnderscore}{\_}

\endinput

% load this file after agda.tex and locallhs2TeX
% also needs amsmath

\usepackage{locallhs2TeX}

\renewcommand{\Varid}[1]{\mathit{#1}}
\renewcommand{\Conid}[1]{\textcolor{AHFunctionColour}{\textsf{#1}}}

% <*> symbol
\newcommand\hap{%
  \ensuremath{\mathbin{\raisebox{0.4pt}{\text{\guilsinglleft}}\mkern-6mu\ast\mkern-6mu\raisebox{0.4pt}{\text{\guilsinglright}}}}%
}

% <|> symbol
\newcommand\halt{%
  \ensuremath{\mathbin{\raisebox{0.4pt}{\text{\guilsinglleft}}\mkern-6mu\mid\mkern-6mu\raisebox{0.4pt}{\text{\guilsinglright}}}}%
}


% <$> symbol
\newcommand\hfmap{%
  \ensuremath{\mathbin{\raisebox{0.4pt}{\text{\guilsinglleft}}\mkern-1mu\$\mkern-1mu\raisebox{0.4pt}{\text{\guilsinglright}}}}%
}

% <$ symbol
\newcommand\hreplace{%
  \ensuremath{\mathbin{\raisebox{0.4pt}{\text{\guilsinglleft}}\mkern-1mu\raisebox{-0.7pt}{\$}}}%
}

% <> symbol
\newcommand\hcmb{%
  \ensuremath{\mathbin{\diamond}}%
}

% <>< symbol
\newcommand\hcmbin{%
  \ensuremath{\rtimes}
}

% <=< symbol
\newcommand\hkcomp{%
  \ensuremath{\mathbin{<\mkern-6.9mu=\mkern-6.77mu<}}
}

% forall 
\newcommand\hforall{%
  \ensuremath{\forall}%
}

% in-style e with !
\newcommand\inunique{%
  \mathrel{\in\mkern-6mu\raisebox{-0.5pt}{!}}
}

\newcommand\ind{\mathbin{!}}

% Anamorphism brackets
\newcommand{\lbparen}{%
  \mathopen{%
    \sbox0{$()$}%
    \setlength{\unitlength}{\dimexpr\ht0+\dp0}%
    \raisebox{-\dp0}{%
      \begin{picture}(.32,1)
        \linethickness{\fontdimen8\textfont3}
        \roundcap
        \put(0,0){\raisebox{\depth}{$($}}
        \polyline(0.32,0)(0,0)(0,1)(0.32,1)
      \end{picture}%
    }%
  }%
}

\newcommand{\rbparen}{%
  \mathclose{%
    \sbox0{$()$}%
    \setlength{\unitlength}{\dimexpr\ht0+\dp0}%
    \raisebox{-\dp0}{%
      \begin{picture}(.32,1)
        \linethickness{\fontdimen8\textfont3}
        \roundcap
        \put(-0.08,0){\raisebox{\depth}{$)$}}
        \polyline(0,0)(0.32,0)(0.32,1)(0,1)
      \end{picture}%
    }%
  }%
}

% Bag brackets
\newcommand{\lbagcon}{\ensuremath{\lbag}}
\newcommand{\rbagcon}{\ensuremath{\rbag}}

% The symbol for mempty
\newcommand{\hmempty}{\Varid{\ensuremath{\epsilon}}}

% A diamond with a line in it
\newcommand{\hmerge}{%
  \ensuremath{\mathbin{\rotatebox[origin=c]{45}{\ensuremath{\boxslash}}}}%
}

\providecommand{\dotdiv}{% Don't redefine it if available
  \mathbin{% We want a binary operation
    \vphantom{+}% The same height as a plus or minus
    \text{% Change size in sub/superscripts
      \mathsurround=0pt % To be on the safe side
      \ooalign{% Superimpose the two symbols
        \noalign{\kern-.35ex}% but the dot is raised a bit
        \hidewidth$\smash{\cdot}$\hidewidth\cr % Dot
        \noalign{\kern.35ex}% Backup for vertical alignment
        $-$\cr % Minus
      }%
    }%
  }%
}

\newcommand{\hmonus}{\dotdiv}

% Properly highlighted N
\newcommand{\hnat}{\Conid{\ensuremath{\mathbb{N}}}}

% Properly highlighted P
\newcommand{\hprob}{\Conid{\ensuremath{\mathbb{P}}}}

% symbol for positive rationals
\newcommand{\hposrat}{\Conid{\ensuremath{\mathbb{Q}^+}}}

% Semiring plus
\newcommand{\hplus}{%
  \ensuremath{\mathbin{\raisebox{0.4pt}{\text{\guilsinglleft}}\mkern-6mu\ensuremath{+}\mkern-6mu\raisebox{0.4pt}{\text{\guilsinglright}}}}%
}

% Semiring times
\newcommand{\htimes}{%
  \ensuremath{\mathbin{\raisebox{0.4pt}{\text{\guilsinglleft}}\mkern-6mu\ensuremath{\cdot}\mkern-6mu\raisebox{0.4pt}{\text{\guilsinglright}}}}%
}

% A one for semirings
\newcommand{\hone}{\numberbb{1}}

% A zero for semirings
\newcommand{\hzero}{\numberbb{0}}

\newcommand{\hnin}{\notin}

\renewcommand{\plus}{\mathbin{+\!\!+}}

% >>- symbol
\newcommand{\fbind}{\mathbin{>\!\!\!>\mkern-8mu-}}

\newcommand{\plussdot}{\mathbin{\dot{\plus}}}
\newcommand{\plussdotdot}{\mathbin{\ddot{\plus}}}


\PrerenderUnicode{í}
%%% The following is specific to ICFP '21 and the paper
%%% 'Algebras for Weighted Search'
%%% by Oisin Kidney and Nicolas Wu.
%%%
\setcopyright{acmlicensed}
\acmPrice{}
\acmDOI{10.1145/3473577}
\acmYear{2021}
\copyrightyear{2021}
\acmSubmissionID{icfp21main-p57-p}
\acmJournal{PACMPL}
\acmVolume{5}
\acmNumber{ICFP}
\acmArticle{72}
\acmMonth{8}


\keywords{Haskell, Agda, graph search, monad}

\begin{CCSXML}
<ccs2012>
   <concept>
       <concept_id>10003752.10003809.10003635.10010037</concept_id>
       <concept_desc>Theory of computation~Shortest paths</concept_desc>
       <concept_significance>500</concept_significance>
       </concept>
   <concept>
       <concept_id>10003752.10003809.10011254.10011255</concept_id>
       <concept_desc>Theory of computation~Backtracking</concept_desc>
       <concept_significance>500</concept_significance>
       </concept>
   <concept>
       <concept_id>10003752.10003790.10003792</concept_id>
       <concept_desc>Theory of computation~Proof theory</concept_desc>
       <concept_significance>300</concept_significance>
       </concept>
   <concept>
       <concept_id>10003752.10003790.10003796</concept_id>
       <concept_desc>Theory of computation~Constructive mathematics</concept_desc>
       <concept_significance>300</concept_significance>
       </concept>
   <concept>
       <concept_id>10003752.10003790.10011740</concept_id>
       <concept_desc>Theory of computation~Type theory</concept_desc>
       <concept_significance>300</concept_significance>
       </concept>
   <concept>
       <concept_id>10003752.10003790.10002990</concept_id>
       <concept_desc>Theory of computation~Logic and verification</concept_desc>
       <concept_significance>300</concept_significance>
       </concept>
   <concept>
       <concept_id>10003752.10010124.10010131.10010132</concept_id>
       <concept_desc>Theory of computation~Algebraic semantics</concept_desc>
       <concept_significance>100</concept_significance>
       </concept>
   <concept>
       <concept_id>10011007.10011006.10011008.10011009.10011012</concept_id>
       <concept_desc>Software and its engineering~Functional languages</concept_desc>
       <concept_significance>500</concept_significance>
       </concept>
   <concept>
       <concept_id>10011007.10011006.10011008.10011024.10011028</concept_id>
       <concept_desc>Software and its engineering~Data types and structures</concept_desc>
       <concept_significance>300</concept_significance>
       </concept>
 </ccs2012>
\end{CCSXML}

\ccsdesc[500]{Theory of computation~Shortest paths}
\ccsdesc[500]{Theory of computation~Backtracking}
\ccsdesc[300]{Theory of computation~Proof theory}
\ccsdesc[300]{Theory of computation~Constructive mathematics}
\ccsdesc[300]{Theory of computation~Type theory}
\ccsdesc[300]{Theory of computation~Logic and verification}
\ccsdesc[100]{Theory of computation~Algebraic semantics}
\ccsdesc[500]{Software and its engineering~Functional languages}
\ccsdesc[300]{Software and its engineering~Data types and structures}
