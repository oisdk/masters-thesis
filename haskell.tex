% load this file after agda.tex and locallhs2TeX
% also needs amsmath

\usepackage{locallhs2TeX}

\renewcommand{\Varid}[1]{\mathit{#1}}
\renewcommand{\Conid}[1]{\textcolor{AHFunctionColour}{\textsf{#1}}}

% <*> symbol
\newcommand\hap{%
  \ensuremath{\mathbin{\raisebox{0.4pt}{\text{\guilsinglleft}}\mkern-6mu\ast\mkern-6mu\raisebox{0.4pt}{\text{\guilsinglright}}}}%
}

% <|> symbol
\newcommand\halt{%
  \ensuremath{\mathbin{\raisebox{0.4pt}{\text{\guilsinglleft}}\mkern-6mu\mid\mkern-6mu\raisebox{0.4pt}{\text{\guilsinglright}}}}%
}


% <$> symbol
\newcommand\hfmap{%
  \ensuremath{\mathbin{\raisebox{0.4pt}{\text{\guilsinglleft}}\mkern-1mu\$\mkern-1mu\raisebox{0.4pt}{\text{\guilsinglright}}}}%
}

% <$ symbol
\newcommand\hreplace{%
  \ensuremath{\mathbin{\raisebox{0.4pt}{\text{\guilsinglleft}}\mkern-1mu\raisebox{-0.7pt}{\$}}}%
}

% <> symbol
\newcommand\hcmb{%
  \ensuremath{\mathbin{\diamond}}%
}

% <>< symbol
\newcommand\hcmbin{%
  \ensuremath{\rtimes}
}

% <=< symbol
\newcommand\hkcomp{%
  \ensuremath{\mathbin{<\mkern-6.9mu=\mkern-6.77mu<}}
}

% forall 
\newcommand\hforall{%
  \ensuremath{\forall}%
}

% in-style e with !
\newcommand\inunique{%
  \mathrel{\in\mkern-6mu\raisebox{-0.5pt}{!}}
}

\newcommand\ind{\mathbin{!}}

% Anamorphism brackets
\newcommand{\lbparen}{%
  \mathopen{%
    \sbox0{$()$}%
    \setlength{\unitlength}{\dimexpr\ht0+\dp0}%
    \raisebox{-\dp0}{%
      \begin{picture}(.32,1)
        \linethickness{\fontdimen8\textfont3}
        \roundcap
        \put(0,0){\raisebox{\depth}{$($}}
        \polyline(0.32,0)(0,0)(0,1)(0.32,1)
      \end{picture}%
    }%
  }%
}

\newcommand{\rbparen}{%
  \mathclose{%
    \sbox0{$()$}%
    \setlength{\unitlength}{\dimexpr\ht0+\dp0}%
    \raisebox{-\dp0}{%
      \begin{picture}(.32,1)
        \linethickness{\fontdimen8\textfont3}
        \roundcap
        \put(-0.08,0){\raisebox{\depth}{$)$}}
        \polyline(0,0)(0.32,0)(0.32,1)(0,1)
      \end{picture}%
    }%
  }%
}

% Bag brackets
\newcommand{\lbagcon}{\ensuremath{\lbag}}
\newcommand{\rbagcon}{\ensuremath{\rbag}}

% The symbol for mempty
\newcommand{\hmempty}{\Varid{\ensuremath{\epsilon}}}

% A diamond with a line in it
\newcommand{\hmerge}{%
  \ensuremath{\mathbin{\rotatebox[origin=c]{45}{\ensuremath{\boxslash}}}}%
}

\providecommand{\dotdiv}{% Don't redefine it if available
  \mathbin{% We want a binary operation
    \vphantom{+}% The same height as a plus or minus
    \text{% Change size in sub/superscripts
      \mathsurround=0pt % To be on the safe side
      \ooalign{% Superimpose the two symbols
        \noalign{\kern-.35ex}% but the dot is raised a bit
        \hidewidth$\smash{\cdot}$\hidewidth\cr % Dot
        \noalign{\kern.35ex}% Backup for vertical alignment
        $-$\cr % Minus
      }%
    }%
  }%
}

\newcommand{\hmonus}{\dotdiv}

% Properly highlighted N
\newcommand{\hnat}{\Conid{\ensuremath{\mathbb{N}}}}

% Properly highlighted P
\newcommand{\hprob}{\Conid{\ensuremath{\mathbb{P}}}}

% symbol for positive rationals
\newcommand{\hposrat}{\Conid{\ensuremath{\mathbb{Q}^+}}}

% Semiring plus
\newcommand{\hplus}{%
  \ensuremath{\mathbin{\raisebox{0.4pt}{\text{\guilsinglleft}}\mkern-6mu\ensuremath{+}\mkern-6mu\raisebox{0.4pt}{\text{\guilsinglright}}}}%
}

% Semiring times
\newcommand{\htimes}{%
  \ensuremath{\mathbin{\raisebox{0.4pt}{\text{\guilsinglleft}}\mkern-6mu\ensuremath{\cdot}\mkern-6mu\raisebox{0.4pt}{\text{\guilsinglright}}}}%
}

% A one for semirings
\newcommand{\hone}{\numberbb{1}}

% A zero for semirings
\newcommand{\hzero}{\numberbb{0}}

\newcommand{\hnin}{\notin}

\renewcommand{\plus}{\mathbin{+\!\!+}}

% >>- symbol
\newcommand{\fbind}{\mathbin{>\!\!\!>\mkern-8mu-}}

\newcommand{\plussdot}{\mathbin{\dot{\plus}}}
\newcommand{\plussdotdot}{\mathbin{\ddot{\plus}}}
