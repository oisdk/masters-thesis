\section{Introduction}
Finiteness in classical mathematics is a simple, almost uninteresting subject.
There are many interesting finite things, mind you, and many of those things are
finite for interesting reasons; finiteness itself, though, is downright boring.

Part of the reason for the simplicity of finiteness is that there's really only
one notion of finiteness: we say that some set is finite if it has a cardinality
equal to some natural number.
If we want to \emph{prove} that something is finite we have at our disposal a
rich array of techniques and insights: we could show that it has a surjection
from another finite set, or that it's equivalent to another finite set.
We could show that any infinite list of its elements must contain duplicates
(this is called ``streamless''), or that there is a finite list which contains
all of its elements.
Once proven, though, all of that complexity collapses.

In constructive mathematics the story is a little different.
When we prove something constructively, often the thing proven retains
components of the proof itself.
Take the proposition \(P\) ``there is a prime number greater than 100''.
\begin{equation}
  P = \exists n. \mathit{isPrime}(n) \wedge n > 100
\end{equation}
A classical proof of the above proposition can avoid naming an \(n\): if we know
there are infinite primes, for instance, it is not too hard to deduce that there
must be some that are larger than \(100\).

The constructive existential, on the other hand, is represented by a concrete
pair.
In the case of the proposition above, the first component of this pair is \(n\),
and the second is a proof that that \(n\) adheres to the predicate
(\(\mathit{isPrime}(n) \wedge n > 100\) in this case).
As such, modulo some caveats we will explain soon, there is no way to provide a
proof of \(P\) which \emph{doesn't} reveal the particular prime number we had in
mind while proving.

This phenomenon is what makes finiteness in constructive mathematics so much
richer and more complex than classical finiteness.
Proving that something is ``streamless'' proves something fundamentally
different to showing a bijection with a finite set.
The other sense in which proofs in constructive mathematics are objects which
can be manipulated is that, when these proofs are written on a computer, they
often can be \emph{executed} as computer programs.
The program ``there is a prime number greater than 100'' isn't a terribly useful
one, but we will see some examples which are later on.

An exploration of finiteness also provides ample opportunity to explain much of
the fundamentals of Homotopy Type Theory \citep{hottbook}: this theory is a
foundational system for mathematics, which has at its core the \emph{univalence
  axiom}.
We will dive much more into this axiom later on, but for now know that it gives
a formal grounding to say ``if two types are isomorphic, it is possible to treat
them as the same.''
We could, for instance, have two different representations for the natural
numbers (binary and unary possibly): the univalence axiom allows us to say ``all
of the things we have proven about the unary numbers are also true about the
binary numbers''.
This is an essential technique in most areas of mathematics, which
had no formal basis in type theory until recently.

All of our work will be formalised in Agda
\citep{norellDependentlyTypedProgramming2008}, a
dependently typed, pure, functional programming language which can be
used as a proof assistant.
Proof assistants are computer programs which verify the correctness of formal
proofs.
Proofs in Agda are programs which can be run: its syntax is quite similar to
Haskell's.
A recent extension to Agda, Cubical Agda
\citep{vezzosiCubicalAgdaDependently2019}, allows Agda to compile and typecheck
programs written in Cubical Type Theory \citep{cohenCubicalTypeTheory2016}: this
type theory gives a \emph{computational} interpretation of the univalence axiom.
By ``computational interpretation'' here we mean that we will be able to run the
programs we write which rely on this axiom.
\subsection{Overview}
In Chapter~\ref{finiteness-predicates} we will look in-depth at the focus of
this paper: finiteness.
As mentioned finiteness in constructive mathematics is a good deal more complex
than finiteness classically:
in this paper alone we will see five separate definitions of ``finiteness'',
each different, and each with reasonable claim to being the ``true'' definition
of finiteness.
We will see which predicates imply each other, what extra information we can
derive from the predicates, and what types are included or excluded from each.
Along the way we will learn a little more about HoTT and univalence, and we'll
see some practical and direct uses of the univalence axiom.

In Chapter~\ref{topos} we will look a little at a slightly more advanced
application of HoTT: topos theory.
A topos is quite a complex, abstract object: it behaves something like the
category of sets, although it is more general.
For us, showing that sets in Homotopy Type Theory form a topos (well, nearly)
will give us access to the general language of toposes.

In Chapter~\ref{search} we will combine everything from the previous sections
into a (somewhat) practical program for proof search.
This program will automatically construct proofs of predicates over finite
domains.
We will demonstrate the program with the countdown problem
\citep{huttonCountdownProblem2002}: this is a somewhat famous puzzle where
players are given five numbers and a target, and have to construct an expression
using some or all of the supplied numbers which evaluates to the target.
\subsection{Contributions}
Since much of this paper is dedicated to exploring well-researched topics from
a new perspective, some of the contributions can be difficult to tease out.
There are three types of valuable and useful work provided in this paper:
\begin{description}
  \item[Exposition of Already-Existing Work]
    Many explorations of dependent types in programming follow the well-trodden
    path of describing length-indexed vectors, explaining one or two aspects of
    totality, and perhaps building up to a proof that \(\mathit{reverse} \circ
    \mathit{reverse} \equiv \mathit{id}\).
    One of the aims of this paper is to provide a different introduction to
    dependent types which also touches on some of the core aspects of HoTT, a
    topic missing much introductory material for functional programmers.
  \item[Formalised Proofs of Theorems proved Informally Elsewhere]
    ``Informal'' here means not machine checked.
    All theorems stated in this paper are proved formally in Cubical Agda and
    machine-checked: the full code for these proofs is available
    online\footnote{\url{https://doisinkidney.com/code/masters-thesis/README.html}}.
    Formal proofs can be quite different from their informal counterparts, and
    it is usually not trivial to formalise a pen-and-paper proof.
    Furthermore, often in formalising a proof new insights about the proof are
    revealed.
  \item[New Fundamental Theoretical Contributions]
    Several of the constructions presented in this paper are in fact novel, and
    some of the theorems proven have not been proven (either formally or
    informally) elsewhere.
\end{description}

What follows is an accounting of the specific technical contributions of this
thesis:
\begin{description}
  \item[A classification of five finiteness predicates in Cubical Type Theory.]
    Four of these predicates have been previously defined and described in
    Homotopy Type Theory \citep{fruminFiniteSetsHomotopy2018}, and two in
    Martin-Löf type theory \citep{fruminFiniteSetsHomotopy2018}, this work defines
    the same predicates in the slightly different setting of Cubical Type
    Theory.
    This has the advantage of giving the proofs computational content, a feature
    used in Section~\ref{search}.
    One of the predicates is new (manifest enumerability), and has not been
    described before.
  \item[Proof of implication between finiteness predicates.]
    Some of these proofs have been seen in the literature before (from
    cardinal finiteness to manifest Bishop finiteness is present informally
    in \cite{yorgeyCombinatorialSpeciesLabelled2014}, and formally (though in
    HoTT) in \cite{fruminFiniteSetsHomotopy2018}).
    \Cref{cardinal-to-manifest-bishop} in particular is new, to the best
    of our knowledge.
  \item[Proof of relation between listed and \(\AgdaDatatype{Fin}\)-based
    finiteness predicates.]
    There are broadly two different ways to define finiteness predicates:
    through a list-based approach, or through relations with a finite prefix of the
    natural numbers.
    Our container-based treatment of the finiteness predicates is novel, to the
    best of our knowledge, and allows us to prove that the two forms of the
    predicates are equivalent, which is also novel.
  \item[Proof that Sets form a \(\Pi\)-Pretopos.]
    This is a proof that has been presented before, in both \cite{hottbook} and
    \cite{rijkeSetsHomotopyType2015}: here we present the first machine-checked
    version of this proof.
  \item[Proof that Kuratowski Finite Sets form a \(\Pi\)-Pretopos
    (\Cref{kuratowski-topos}).]
    This is a new proof, both formally and informally.
  \item[A library for proof search.]
    While several libraries for proof search based on finiteness exist in
    dependently-typed programming languages
    \citep{firsovDependentlyTypedProgramming2015}, the one that is presented
    here is strictly more powerful than those that came before as it is capable
    of including functions in the search space.
    This is a direct consequence of the use of Cubical Type Theory: in MLTT
    functions would not be able to be included at all, and in HoTT the
    univalence axiom would not have computational content, so a proof search
    library (structured in the way presented here) would not function.
    There are other more minor contributions of this library: they are described
    in detail in subsection~\ref{search}.
  \item[A verified solver for the countdown problem.]
    The countdown problem \citep{huttonCountdownProblem2002} has been studied
    from a number of angles in functional programming
    \cite{birdCountdownCaseStudy2005}; ours is the first verified solver of the
    problem.
\end{description}
\begin{comment}

\subsection{What is a Proof?}
Constructive mathematics is, fundamentally, a different way of thinking about
proofs.
Classically we have two different universes of things: we have
objects, like the natural numbers, or the rings; and the proofs on those things.
Constructive mathematics joins those two worlds together.

Practically speaking this means that in constructive mathematics we can't
always ``prove the existence of'' some thing; instead we simply provide that
thing.
We mentioned an example where this comes into play: to prove that something is
finite we might want to prove that it is in bijection with some other finite
type.
Constructively, this proof \emph{is} a bijection.

Constructive mathematics is also deeply tied to computers.
Dependently-typed programming languages, like Agda
\citep{norellDependentlyTypedProgramming2008}, Coq
\citep{thecoqdevelopmentteamCoqProofAssistant2020}, or Idris
\citep{bradyIdrisGeneralpurposeDependently2013}, allow us to compile and
\emph{run} our proofs.
If we provide a bijection between two types in Agda we also provide a function
to and from each type.
Agda is therefore both a formal language for constructive proofs and a
programming language which can run those proofs.

This relation between proofs and programs is often called the Curry-Howard
correspondence.
Under this framework, propositions in the logical sense correspond to types in
the programming sense \citep{wadlerPropositionsTypes2015}; proofs of those
propositions then correspond to programs which inhabit those types.
As an example we have from propositional logic the following:
\begin{equation*}
  \frac{A \wedge B}{A}
\end{equation*}
This is conjunction elimination: if we know \(A\) and \(B\), we also know \(A\).

Via Curry-Howard, the type corresponding to conjunction is in fact the pair (or
tuple).
\(A\) and \(B\) are also types.
So, the above proposition, as a type, is the following:
\begin{equation*}
  \mathit{fst}\mathbin{::}(a,b)\to a
\end{equation*}
And the proof which inhabits this type?
It is the function often called \verb+fst+: the selector for the first component
of the pair.
\begin{equation*}
  \mathit{fst}\;(x,y) = x
\end{equation*}

Occasionally we don't want to provide the extra information that
constructiveness demands: it is occasionally useful to say ``there exists some
\(x\) which satisfies some predicate \(P\)'' without revealing the value of \(x\).
In other words, we want to retrieve something of the classical notion of an
existential in a constructive setting.
Later, for instance, we will begin to work with the
category of finite sets.
The objects of this category are (of course) finite sets: however, if our notion
of ``finite'' is tied to an explicit bijection, that means that the type of
``things which are finite'' is pairs of types and bijections between those
types. 
Since there are \(n!\) bijections between a type of cardinality \(n\) and any
other, this means that for every such type we have \(n!\) different objects,
instead of just 1.
We'll see a way to fix this problem with Homotopy Type Theory.

The proofs we will provide in this paper will be written in the syntax of
Agda: they are, however, all valid classical proofs.
Constructive mathematics is a subset of classical, after all.
When we say ``constructive'' we really just mean that these proofs avoid
reliance on certain axioms like the law of the excluded middle, or the axiom of
choice.
\end{comment}
\subsection{Homotopy Type Theory}
HoTT is both a type theory and a foundational theory for mathematics.
As a type theory it can be thought of as an alternative to things like
Martin-Löf type theory; as a foundational theory it can be compared to other
foundational theories like Zermelo–Fraenkel set theory.
\begin{comment}
\begin{definition}[ZFC]
  Zermelo-Fraenkel set theory with choice is a foundational theory for
  mathematics, which is based on set theory.
  It is the most commonly-used foundational theory (although it is important to
  stress that most modern mathematics is foundation-agnostic).

  The Axiom of Choice is a non-constructive axiom which is independent of the
  rest of Zermelo-Fraenkel set theory: when included the theory is often
  abbreviated to ``ZFC''.

  A full exploration of ZFC and its axioms is beyond the scope of this paper;
  we will, see, however, a definition of the axiom of choice in CuTT
  (\Cref{axiom-of-choice}).
\end{definition}
\end{comment}

Central to the theory is the univalence axiom.
This axiom states that isomorphism implies equality: more precisely, that
``equivalence'' is equivalent to equality.
We haven't defined equivalence yet (and we haven't defined equality rigorously
yet), but the core thrust of the axiom is that it gives proofs of isomorphism
all the power of proofs of equality.

From a proof perspective, this is quite useful for transporting proofs from one
domain to another: for instance, a proof that some type \(A\) is finite can be
transported to a proof that any type isomorphic to \(A\) is finite.
From a programming perspective, this is quite useful for transporting
\emph{programs}: an API defined on some type can suddenly be reused, without
change, on another type as long as the latter is isomorphic to the former.
It's worth noting that the technique mentioned here is used quite
pervasively in everyday mathematics: it's just that foundational systems (like
MLTT) could not justify its use.

The other addition that HoTT gives to traditional type theory is that of Higher
Inductive Types.
Traditional inductive types are defined by listing their constructors: with HITs
we can also list the equalities they must satisfy.
The most obvious immediate use of this is quotient types: another concept which
is regarded as standard in normal mathematics, but somewhat embarrassingly
missing from most constructive theories.
In truth, the fact that HITs can be used to describe simple set quotients is
almost incidental: in HoTT they are a far more powerful, general tool,
which can be used to define a wide variety of essential constructions like the
circle, torus, etc.
This paper isn't very interested in these homotopy-focused topics, however it
is worth mentioning as it is one of HoTT's great strengths.

Finally, we have to explain where Cubical Type Theory
\citep{cohenCubicalTypeTheory2016} fits in.
Strictly speaking in this paper we will not work directly in HoTT: we work
instead in CuTT, which is very closely related, but not quite the same, as HoTT.
CuTT's main departure from HoTT is in its representation of equalities, which it
calls ``paths''.
In HoTT (and Martin-Löf type theory, which we will describe later) equalities
are an inductive type.
In CuTT, paths behave like functions from the real line: an equality
\(x \equiv y\) is a function which, when applied to \(0\), returns \(x\), and
when applied to \(1\), returns \(y\).
\begin{equation}
\begin{aligned}
  & e  :    & x \equiv y \\
  & e(i) = & \begin{cases}
    x, & i = 0 \\
       & \vdots \\
    y, & i = 1
  \end{cases}
\end{aligned}
\end{equation}
This different ``implementation'' of paths is central to how CuTT can give
computational content to univalence, but it also slightly changes the way paths
function.

There is one crucial difference between the two theories: in CuTT the
univalence ``axiom'' is in fact a theorem, with computational content.
This means that univalence is not a built-in, assumed to be true, but instead
it's actually derived from other axioms in the system.
Practically speaking it means that univalence has computational content: i.e. if
we see that two types are equal then we can actually derive the isomorphism that
that equality implies.
As we're using CuTT as a programming language (in the form of Cubical Agda
\citep{vezzosiCubicalAgdaDependently2019}), this is an essential feature.
It means that the proofs we write using univalence still correspond to programs
which can be run.

\footnotetext{
  To be absolutely correct we should not say that CuTT implements HoTT, as the
  two have some subtle differences.
  While everything we can prove in HoTT can also be proven in CuTT,
  it is not the case that one is a strict subset of the other.
  CuTT has many of the same features of HoTT, like HITs and univalence, so
  almost all of the theory of HoTT applies to both (and indeed almost all of the
  univalent foundations program applies to both), but there are some slight
  differences between the theories which mean that some proofs which are valid
  HoTT are not valid in CuTT.
}

