\section{Introduction}
Unlike its classical counterpart, finiteness in constructive mathematics is a
rich and complex subject.
Classically, something is ``finite'' if it has a cardinality equal to some
natural number; every notion of finiteness is equivalent to this one.
Constructively, however, there are multiple distinct notions of finiteness, each
with interesting properties~\citep{coquandConstructivelyFinite2010,
  firsovVariationsNoetherianness2016, fruminFiniteSetsHomotopy2018}.

Part of the reason for different notions of finiteness in constructive
mathematics is that constructive proofs are ``leaky''; as mathematical
objects, they tend to reveal some information about \emph{how} they were proven,
where a proof of a classical proposition only reveals that the proposition is
true.
The \verb+Listable+ predicate
from~\citet{firsovDependentlyTypedProgramming2015}, for instance (called Split
Enumerable (\Cref{split-enum-def}) in this paper),
states that a type is finite if every element of the type is present in some
finite list.
A proof of this predicate, then, will reveal not just that the type in question
is finite, but it will also specify a total order.
And different proofs of \verb+Listable+ that use different total orders will be
distinguishable.

Recent developments in type
theory~\citep[in particular Cubical Agda][]{vezzosiCubicalAgdaDependently2019}
have made constructions like propositional truncation available in proof
assistants, which
can be used to patch this ``leak'': any two propositionally truncated values of
the same type are indistinguishable.
The propositional truncation of \verb+Listable+, called Cardinal
finiteness~(\Cref{card-finite-def}), does not reveal an ordering, for instance.
However, far from closing the gap between classical and constructive proofs,
this only makes the divide more subtle: while we can't extract quantities like a
total order from a proof of cardinal finiteness, we can extract computational
content like an algorithm for decidable equality
(\Cref{cardinal-finite-discrete}), or the cardinality of the underlying type
(\Cref{card-finite-cardinality}).

This paper will catalogue five separate variants of finiteness in constructive
mathematics, and study their relationships and properties.
All of our work will be formalised in Agda
\citep{norellDependentlyTypedProgramming2008}, a
dependently typed, pure, functional programming language which can be
used as a proof assistant.
A recent extension to Agda, Cubical Agda
\citep{vezzosiCubicalAgdaDependently2019}, allows Agda to compile and typecheck
programs written in Cubical Type Theory \citep{cohenCubicalTypeTheory2016}: this
type theory gives a \emph{computational} interpretation of the univalence axiom
\citep{hottbook}.
By ``computational interpretation'' here we mean that we will be able to run the
programs we write which rely on this axiom.
The univalence axiom allows us to do Homotopy Type Theory (HoTT) in Agda.

We will use this computational capability to build a library for proof search
for predicates over finite domains.
This proof search library will be further used to build a verified solver for
the countdown problem.

\subsection{Overview}
In Section~\ref{finiteness-predicates} we will look in-depth at the focus of
this paper: finiteness.
As mentioned finiteness in constructive mathematics is a good deal more complex
than finiteness classically:
in this paper alone we will see five separate definitions of ``finiteness''.
We will see which predicates imply each other, what extra information we can
derive from the predicates, and what types are included or excluded from each.
Along the way we will learn a little more about HoTT and univalence, and we'll
see some direct uses of the univalence axiom.

In Section~\ref{topos} we will look a little at a slightly more advanced
application of HoTT:\@ topos theory.
A topos is quite a complex, abstract object: it behaves something like the
category of sets, although it is more general.
We will show that finite sets in HoTT form a $\Pi$-pretopos.

In Section~\ref{search} we will present a library for finite proof search.
We will demonstrate the library with the countdown problem
\citep{huttonCountdownProblem2002}: this is a puzzle where
players are given five numbers and a target, and have to construct an expression
using some or all of the supplied numbers which evaluates to the target.

\subsection{Contributions}
\begin{description}
  \item[A classification of five finiteness predicates in Cubical Type Theory.]
    Four of these predicates have been previously defined and described in
    Homotopy Type Theory \citep{fruminFiniteSetsHomotopy2018}, and two in
    Martin-Löf type theory \citep{fruminFiniteSetsHomotopy2018}, this work defines
    the same predicates in the slightly different setting of Cubical Type
    Theory.
    This has the advantage of giving the proofs computational content, a feature
    used in Section~\ref{search}.
  \item[Proof of implication between finiteness predicates.]
    Some of these proofs have been seen in the literature before (from
    cardinal finiteness to manifest Bishop finiteness is present informally
    in~\cite{yorgeyCombinatorialSpeciesLabelled2014}, and formally (though in
    HoTT) in~\cite{fruminFiniteSetsHomotopy2018}).
    \Cref{cardinal-to-manifest-bishop} in particular is new, to the best
    of our knowledge.
  \item[Proof of relation between listed and \(\AgdaDatatype{Fin}\)-based
    finiteness predicates.]
    There are broadly two different ways to define finiteness predicates:
    through a list-based approach, or through relations with a finite prefix of the
    natural numbers.
    Our container-based treatment of the finiteness predicates  allows us to
    prove that the two forms of the predicates are equivalent.
  \item[Proof that Sets form a \(\Pi\)-Pretopos.]
    This is a proof that has been presented before, in both~\citet{hottbook}
    and~\citet{rijkeSetsHomotopyType2015}: here we present a machine-checked
    version of this proof.
  \item[Proof that Kuratowski Finite Sets form a \(\Pi\)-Pretopos
    (\Cref{kuratowski-topos}).]
  \item[A library for proof search.]
    While several libraries for proof search based on finiteness exist in
    dependently-typed programming languages
    \citep{firsovDependentlyTypedProgramming2015}, the one that is presented
    here is strictly more powerful than those that came before as it is capable
    of including functions in the search space.
    This is a direct consequence of the use of Cubical Type Theory: in MLTT
    functions would not be able to be included at all, and in HoTT the
    univalence axiom would not have computational content, so a proof search
    library (structured in the way presented here) would not function.
    There are other more minor contributions of this library: they are described
    in detail in subsection~\ref{search}.
  \item[A verified solver for the countdown problem.]
    The countdown problem \citep{huttonCountdownProblem2002} has been studied
    from a number of angles in functional
    programming~\cite{birdCountdownCaseStudy2005}; our solver is the first 
    machine-checked solver, to the best of our knowledge.
\end{description}
