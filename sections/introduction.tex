\chapter{Introduction}


% \begin{agdalisting}
%   \ExecuteMetaData[agda/Snippets/Interactive.tex]{step1}
% \end{agdalisting}
% \begin{agdalisting}
%   \ExecuteMetaData[agda/Snippets/Interactive.tex]{step2}
% \end{agdalisting}
% \begin{agdalisting}
%   \ExecuteMetaData[agda/Snippets/Interactive.tex]{step3}
% \end{agdalisting}

% \begin{equation}
%   \frac{
%     \mathit{zs}:\AgdaDatatype{List}\;A \;\;\;\;
%     \mathit{ys}:\AgdaDatatype{List}\;A \;\;\;\;
%     A :\AgdaDatatype{Type}\;a\;\text{(not in scope)} \;\;\;\;
%     a :\AgdaDatatype{Level}\;\text{(not in scope)}
%     }{
%     \mathit{ys}\;\AgdaFunction{++}\;\mathit{zs}
%     \;\AgdaFunction{≡}\;
%     \mathit{ys}\;\AgdaFunction{++}\;\mathit{zs}
%   }
% \end{equation} 

This thesis will explore and explain the notion of finiteness in constructive
mathematics: using this setting, it will also serve as an introduction to
constructive mathematics in Cubical Agda, and some related topics.

In constructive mathematics proofs are much more substantial things than their
classical counterparts.
A constructive proof is a mathematical object that can be manipulated in the
same way that an integer or an algorithm can: indeed, some of the ``proofs'' we
will see in this thesis will emerge to be useful programs in disguise.

Finiteness is one of the classic examples of a topic for which the constructive
interpretation is far more complex and involved than the classical.
In classical mathematics, to say something is ``finite'' amounts to
demonstrating that an appropriate bijection or surjection exists.


the setting
of Cubical Type Theory~\cite{cohenCubicalTypeTheory2016}.
It also provides ample opportunity to demonstrate the computational properties
of constructive proofs: in developing a theory for finiteness we will also
develop a program which can search for verified solutions to the countdown
problem~\cite{huttonCountdownProblem2002}. \todo{find proper cite for this}
In fact we will prove that, given a particular numbers puzzle, there are only
finitely many candidate solutions: this proof is precisely the program which can
search for and list these solutions.

This thesis is aimed at individuals with some knowledge of Haskell (although
extensive knowledge of Haskell is not an absolute necessity) and a curiosity
about dependent types.
We will explain the basics of dependently-typed programming in Agda, how to
write programs and how to write proofs, and we will explain something of the
internals of dependent type theory along the way.
One way we depart from the usual introduction to dependent types is that we will
use Cubical Type Theory from the very beginning: this rather new type theory has
an implementation in the form of Cubical Agda
\cite{vezzosiCubicalAgdaDependently2019}, which is what we will use for the
actual code throughout.
CuTT provides a setting for Homotopy Type Theory \cite{hottbook}, a new
foundational approach to mathematics.
No knowledge of either of these topics is assumed:
as it happens, finiteness touches upon many of the basics of HoTT quite
naturally, so we will have opportunities to explain much of the topic as we go.

\section{Overview}
This thesis is structured as follows: in Chapter~\ref{agda-intro} we will
introduce Agda, its syntax, and quickly bring reader up to speed with how to
write programs in it.
We will also begin to talk about some more foundational type-theory concepts,
and we will explain a little about Agda's particular interpretation of HoTT and
CuTT.

In Chapter~\ref{finiteness-predicates} we will look in-depth at the focus of
this thesis: finiteness.
We will see that there is no one ``true'' notion of ``finite'', and we will
delineate and compare the competing definitions.
Along the way we will learn a little more about HoTT and univalence, and we'll
see some practical and direct uses of the univalence axiom.

In Chapter~\ref{topos} we will look a little at a slightly more advanced
application of HoTT: topos theory.
Here we will see some of the strengths of HoTT and CuTT, as they allow us to
work in certain mathematical settings inaccessible to MLTT.

In Chapter~\ref{search} we will combine everything from the previous chapters
into a (somewhat) practical program for proof search.
Here is where we will develop the countdown solver.

Finally, in Chapter~\ref{infinite} we will see how we can adapt the work of
previous chapters to the setting of countably infinite types.
\section{What is a Proof?}
\section{Homotopy Type Theory}

% We are interested in constructive notions of finiteness, formalised in Cubical
% Type Theory
% In this paper we will explore five such notions of finiteness, including their
% categorical interpretation, and use them to build a simple proof-search library
% facilitated in a fundamental way by univalence.
% Along the way we will use the Countdown
% problem as an example, and provide a program
% which produces verified solutions to the puzzle.
% We will also briefly examine countability, and demonstrate its parallels and
% differences with finiteness.
% \section{The Varieties of Finiteness}
% \todo{Make all references parenthetical}
% In Section~\ref{finiteness-predicates} we will explore a number of different
% predicates for finiteness.
% In contrast to classical finiteness, in a constructive setting there are many
% predicates which all have some claim to being the formal interpretation of
% ``finiteness''~\cite{coquandConstructivelyFinite2010}.
% The particular predicates we are interested in are organised in
% Figure~\ref{finite-classification}: each arrow in the diagram represents a proof
% that one predicate can be derived from another.
% Each arrow in Figure~\ref{finite-classification} corresponds to a proof of
% implication: cardinal finiteness, for instance, with a strict total order,
% implies split enumerability (Theorem~\ref{cardinal-to-manifest-bishop}).


% These finiteness predicates differ along two main axes: informativeness, and
% restrictiveness.
% More ``informative'' predicates have proofs which contain extraneous information
% other than the finiteness of the underlying type: a proof of split enumerability
% (Section~\ref{split-enumerability}), for instance, comes with a strict total
% order on the underlying type.

% The ``restrictiveness'' of a predicate refers to how many types it admits into
% its notion of ``finite''.
% There are strictly more Kuratowski finite (Section~\ref{kuratowski}) types than
% there are Cardinally finite (Section~\ref{cardinal-finiteness}).

% Proofs coming with extra information is a common theme in constructive
% mathematics: often this extra information is in the form of an algorithm which
% can do something useful related to the proof itself.
% Indeed, our proofs of finiteness here will provide an algorithm to solve the
% countdown puzzle.
% Occasionally, however, the extra information is undesirable: we may want to
% assert the existence of some value \(x : A\) which satisfies a predicate \(P\)
% without revealing \emph{which} \(A\) we're referring to.
% More concretely, we will need in this paper to prove that two types are in
% bijection without specifying a particular bijection.
% This facility is provided by Homotopy Type Theory~\cite{hottbook} in the form of
% propositional truncation, and it is what allows us to prove the bulk of
% propositions in this paper.

% For each predicate we will also prove its closure properties (i.e.\ that the
% product of two finite sets is finite).
% The most significant of these closure proofs is that of closure under \(\Pi\)
% (dependent functions) (Theorem~\ref{split-enum-pi}).
% \section{Toposes and Finite Sets}
% In Section~\ref{topos}, we will explore the categorical interpretation of
% decidable Kuratowski finite sets.
% The motivation here is partially a practical one: by the end of this work we
% will have provided a library for proof search over finite types, and the
% ``language'' of a topos is a reasonable choice for a principled language for
% constructing proofs of finiteness in the style of
% QuickCheck~\cite{claessenQuickCheckLightweightTool2011} generators.

% Theoretically speaking, showing that sets in Homotopy Type Theory form a topos
% (with some caveats) is an important step in characterising the categorical
% implications of Homotopy Type Theory, first proven
% in~\cite{rijkeSetsHomotopyType2015}. \todo{This reference should be citet not citep}
% Our work is a formalisation of this result (and the first such formalisation
% that we are aware of).
% The proof that decidable Kuratowski finite sets form a \(\Pi\)-pretopos is
% additional to that.
% \section{Countability Predicates}
% After the finite predicates, we will briefly look at the infinite countable
% types, and classify them in a parallel way to the finite predicates
% (Section~\ref{infinite}).
% We will see that we lose closure under function arrows, but we gain it under the
% Kleene star (Theorem~\ref{split-countability-sigma}).
% \section{Search}
% All of our work is formalised in Cubical
% Agda~\cite{vezzosiCubicalAgdaDependently2019}: as a result, the constructive
% interpretation of each proof is actually a program which can be run on a
% computer.
% In finiteness in particular, these programs are particularly useful for
% exhaustive search.

% We will use the countdown problem as a running example throughout the paper: we
% will show how to prove that any given puzzle has a finite number of solutions,
% and from that we will show how to enumerate those solutions, thereby solving the
% puzzle in a verified way.

% In Section~\ref{search} we will package up the ``search'' aspect of finiteness
% into a library for proof search: similar libraries have been built
% in~\cite{fruminFiniteSetsHomotopy2018}
% and~\cite{firsovDependentlyTypedProgramming2015}.
% Our library differs from those in three important ways: firstly, it is strictly
% more powerful, as it allows for search over function types.
% Secondly, finiteness proofs also provide equivalence proofs to any other finite
% type: this allows transport of proofs between types of the same cardinality.
% Finally, through generic programming we provide a simple syntax for stating
% properties which mimics that of QuickCheck.
% We also ground the library in the theoretical notions of omniscience.
% \section{Notation and Background}
% We work in Cubical Type Theory~\cite{cohenCubicalTypeTheory2016}, specifically
% Cubical Agda~\cite{vezzosiCubicalAgdaDependently2019}.
% Cubical Agda is a dependently-typed functional programming language, based on
% Martin-Löf Intuitionistic Type Theory, with a Haskell-like syntax.

% Being a dependently-typed language, we'll have to be clear about what we mean
% when we say ``type'' in Agda.
% \begin{definition}[Type]
%   We use \(\AgdaDatatype{Type}\) to denote the universe of (small) types.
%   The universe level is denoted with a subscript number, starting at 0.
%   ``Type families'' are functions into \(\AgdaDatatype{Type}\).
% \end{definition}

% The are two broad ways to define types in Agda: as an inductive
% \(\AgdaKeyword{data}\) type, similar to data type definitions in Haskell, or as
% a \(\AgdaKeyword{record}\).
% Here we'll define the basic type formers used in MLTT.\@
% \begin{definition}[Basic Types]
%   The three basic types---often called 0, 1, and 2 in MLTT---here will be
%   denoted with their more common names: \(\bot\), \(\top\), and
%   \(\mathbf{Bool}\), respectively.
%   \begin{multicols}{3}
%     \begin{agdalisting*}
%       \ExecuteMetaData[agda/Snippets/Introduction.tex]{bot}
%     \end{agdalisting*} \columnbreak
%     \begin{agdalisting*}
%       \ExecuteMetaData[agda/Snippets/Introduction.tex]{top}
%     \end{agdalisting*} \columnbreak
%     \begin{agdalisting*}
%       \ExecuteMetaData[agda/Snippets/Introduction.tex]{bool}
%     \end{agdalisting*}
%   \end{multicols}
% \end{definition}
% \begin{definition}[The Dependent Sum]
%   Dependent sums are denoted with the usual \(\Sigma\) symbol, and has the
%   following definition in Agda:

%   \begin{agdalisting*}
%     \ExecuteMetaData[agda/Snippets/Introduction.tex]{sigma}
%   \end{agdalisting*}
%   We will use different notations to refer to this type depending on the
%   setting.
%   The following four expressions all denote the same type:

%   \begin{multicols}{4}
%     \begin{agdalisting*}
%       \ExecuteMetaData[agda/Snippets/Introduction.tex]{sigma-syntax-1}
%     \end{agdalisting*} \columnbreak
%     \begin{agdalisting*}
%       \ExecuteMetaData[agda/Snippets/Introduction.tex]{sigma-syntax-3}
%     \end{agdalisting*} \columnbreak
%     \begin{agdalisting*}
%       \ExecuteMetaData[agda/Snippets/Introduction.tex]{sigma-syntax-4}
%     \end{agdalisting*} \columnbreak
%     \begin{agdalisting*}
%       \ExecuteMetaData[agda/Snippets/Introduction.tex]{sigma-syntax-2}
%     \end{agdalisting*}
%   \end{multicols} \vspace{-1\baselineskip} \noindent
%   The non-dependent product is a special instance of the dependent.
%   We denote a simple pair of types \(A\) and \(B\) as \(A \times B\).
% \end{definition}
% \begin{definition}[Dependent Product]
%   Dependent products (dependent functions) use the \(\Pi\) symbol.
%   The three following expressions all denote the same type:
%   \begin{multicols}{3}
%     \begin{agdalisting*}
%       \ExecuteMetaData[agda/Snippets/Introduction.tex]{pi-syntax-1}
%     \end{agdalisting*} \columnbreak
%     \begin{agdalisting*}
%       \ExecuteMetaData[agda/Snippets/Introduction.tex]{pi-syntax-2}
%     \end{agdalisting*} \columnbreak
%     \begin{agdalisting*}
%       \ExecuteMetaData[agda/Snippets/Introduction.tex]{pi-syntax-3}
%     \end{agdalisting*}
%   \end{multicols}\vspace{-1.5\baselineskip}\noindent
%   Non-dependent functions are denoted with the arrow (\(\rightarrow\)).
% \end{definition}

% At this point, as a quick example, we can define the first of our objects for
% the countdown transformation: the vector of Booleans for selection.
% A vector is relatively simple to define: a vector of zero elements is simply a
% unit, a vector of \(n+1\) elements is the product of an element and a vector of
% \(n\) elements.
% \begin{agdalisting*}
%   \ExecuteMetaData[agda/Data/Vec/Iterated.tex]{vec-def}
% \end{agdalisting*}
% From this we can see that a vector of \(n\) Booleans has the type
% \(\AgdaDatatype{Vec} \; \AgdaDatatype{Bool} \; n\)

% Finally, there is one last thing we must define before moving on to the
% finiteness predicates: paths.
% \begin{definition}[Path Types]\label{path-types}
%   The equality type (which we denote with \(\equiv\)) in CuTT is the type of
%   Paths\footnotemark.
%   The nature and internal structure of Paths is complex and central to how
%   Cubical Type Theory ``implements'' Homotopy Type Theory, but those details are
%   not relevant to us here.
%   Instead, we only need to know that univalence holds for paths, and path types
%   do indeed compute in Cubical Agda.
% \end{definition}

% \footnotetext{%
%   Actually, CuTT does have an identity type with similar semantics to the
%   identity type in MLTT.\@
%   We do not use this type anywhere in our work, however, so we will not consider
%   it here.
% }

%%% Local Variables:
%%% mode: latex
%%% TeX-master: "../paper"
%%% End:
