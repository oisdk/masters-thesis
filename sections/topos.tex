\section{Topos}\label{topos}
In this subsection we will examine the categorical interpretation of finite
sets.
In particular, we will prove that discrete Kuratowski finite types form a
\(\Pi\)-pretopos.

\subsection{Categories in HoTT}
In this subsection we will present the encoding of (pre) categories in HoTT.
Much of this subsection is simply a summary of parts of \citet[section
9]{hottbook}.
The formal proofs we provide are part translation of those proofs in that
section, part from \citep{iversenFredefoxCat2018}
\citep{huProofrelevantCategoryTheory2020}, and part our own.

While it is possible to use HoTT to do some category theory, there is not (at
present) a fully-general encoding of categories that avoids issues with higher
homotopy structure.
As a result, we will encode \emph{pre}categories: a precategory is a category
such that all of the arrows are sets.

\subsection{The Category of Sets}
Next we'll look at how to construct the category of sets (in the HoTT sense).
Much of this work comes directly from \citet{rijkeSetsHomotopyType2015} and
\citet[section 10]{hottbook} (the latter of which is in fact an updated and
slightly less detailed version of the former).
In particular, our treatment (and definition) of categories and topoi comes
directly from those works.
We have provided in the formalisation a proof that sets in CuTT form a
\(\Pi\)-pretopos: this proof (in HoTT) is in fact the main result of
\citet{rijkeSetsHomotopyType2015}; our contribution is simply the formalisation.

The objects of this category are represented by a \(\Sigma\):
\ExecuteMetaData[agda/Snippets/Category.tex]{hset}

This will be quite similar to our objects for finite sets.

Since sets in HoTT don't form a topos, there are quite a few smaller lemmas we
need to prove to get as close as we can (a \(\Pi W\)-pretopos): we won't include
them here, other than the closure proofs in the following subsection.
\subsection{Closure}
The two most involved proofs for showing that discrete Kuratowski sets form a
\(\Pi\)-pretopos are those proofs that show closure under \(\Pi\) and
\(\Sigma\).
We will describe them here.

\paragraph{Closure of the Ordered Predicates}
First, we will show that split enumerability (and, by extension, manifest
enumerability) are closed under \(\Pi\) and \(\Sigma\).

\begin{lemma}\label{split-enum-sigma}
  Split enumerability is closed under \(\Sigma\).
  \ExecuteMetaData[agda/Cardinality/Finite/SplitEnumerable.tex]{split-enum-sigma}
\end{lemma}
\begin{proof}
  Our task is to construct the two components of the output pair: the support
  list, and the cover proof.
  We'll start with the support list: this is constructed by taking the Cartesian
  product of the input support lists.
  \begin{multicols}{2}
    \ExecuteMetaData[agda/Cardinality/Finite/SplitEnumerable.tex]{sup-sigma}
  \end{multicols}
  We use do notation here because we're working the list monad: this applies the
  function \(ys\) to every element of the list \(xs\), and concatenates
  the results.
  The cover proof is in our formalisation.
\end{proof}

Next we'll look at closure under \(\Pi\).
Our proof here makes use directly of the univalence axiom, and makes use
furthermore of the previous closure proof.
\begin{theorem}\label{split-enum-pi}
  Split enumerability is closed under dependent functions
  (\(\Pi\)-types).
  \ExecuteMetaData[agda/Cardinality/Finite/ManifestBishop.tex]{pi-clos}
\end{theorem}
\begin{proof}
  Let \(A\) be a split enumerable type, and \(U\) be a type family from \(A\),
  which is split enumerable over all points of \(A\).

  As \(A\) is split enumerable, we know that it is also manifestly Bishop finite
  (\Cref{split-enum-to-manifest-bishop}), and consequently we know \(A
  \simeq \AgdaDatatype{Fin}\;n\), for some \(n\) (\Cref{bishop-equiv}).
  We can therefore replace all occurrences of \(A\) with \(\AgdaDatatype{Fin}\;n\),
  changing our goal to:
  \begin{equation*}
    \frac{
      \AgdaDatatype{\ensuremath{\mathcal{E}!}}\;(\AgdaDatatype{Fin}\;n) \; \; \; \left((x : \AgdaDatatype{Fin}\;n) \rightarrow \AgdaDatatype{\ensuremath{\mathcal{E}!}}\;\left( U\;x \right)\right)
    }{
      \AgdaDatatype{\ensuremath{\mathcal{E}!}}\left((x : \AgdaDatatype{Fin}\;n) \rightarrow U\;x\right)
    }
  \end{equation*}
  
  We then define the type of \(n\)-tuples over some type family.
  
    \ExecuteMetaData[agda/Data/Tuple/UniverseMonomorphic.tex]{tuple-def}
  
  We can show that this type is equivalent to functions (proven in our formalisation):
  
    \ExecuteMetaData[agda/Data/Tuple/UniverseMonomorphic.tex]{tuple-iso}
  
  And therefore we can simplify again our goal to the following:
  \begin{equation*}
    \frac{
      \AgdaDatatype{\ensuremath{\mathcal{E}!}}\;(\AgdaDatatype{Fin}\;n) \; \; \; ((x : \AgdaDatatype{Fin}\;n) \rightarrow \AgdaDatatype{\ensuremath{\mathcal{E}!}}\left( U\;x \right))
    }{
      \AgdaFunction{\ensuremath{\mathcal{E}!}}\;\left(\AgdaFunction{Tuple}\;n\;U\right)
    }
  \end{equation*}
  
  We can prove this goal by showing that \(\AgdaFunction{Tuple}\;n\;U\) is split
  enumerable: it is made up of finitely many products of points of \(U\), which
  are themselves split enumerable, and \agdatop, which is also split enumerable.
  \Cref{split-enum-sigma} shows us that the product of finitely many split
  enumerable types is itself split enumerable, proving our goal.
\end{proof}
\paragraph{Closure on Cardinal Finiteness}
Since we don't have a function of type \(\agdacal{C}\;A \rightarrow
\AgdaDatatype{\ensuremath{\mathcal{B}}}\;A\), closure proofs on \(\AgdaDatatype{\ensuremath{\mathcal{B}}}\) do not transfer over to
\agdacal{C} trivially (unlike with \(\AgdaDatatype{\ensuremath{\mathcal{E}!}}\) and \(\AgdaDatatype{\ensuremath{\mathcal{B}}}\)).
The cases for \agdabot, \agdatop, and \(\AgdaDatatype{Bool}\) are simple to adapt: we
can just propositionally truncate their Bishop finiteness proof.

Non-dependent operators like \AgdaFunction{\(\times\)},
\AgdaFunction{\(\uplus\)}, and \(\rightarrow\) are also relatively
straightforward: since \(\AgdaDatatype{\ensuremath{\lVert\_\rVert}}\) forms a
monad, we can apply \(n\)-ary functions to values inside it, combining them
together.

However, for the dependent type formers like \(\Sigma\) and \(\Pi\), the
same trick does not work.
We have closure proofs like:
\begin{equation*}
  \frac{
    \AgdaDatatype{\ensuremath{\mathcal{B}}}\;A \; \; \; ((x : A) \rightarrow \AgdaDatatype{\ensuremath{\mathcal{B}}}\;(U\;x))
  }{
    \AgdaDatatype{\ensuremath{\mathcal{B}}}\;((x : A) \rightarrow U\;x)
  }
\end{equation*}
If we apply the monadic truncation trick we can derive closure proofs like the
following:
\begin{equation*}
  \frac{
    \AgdaDatatype{\ensuremath{\lVert}}\; \AgdaDatatype{\ensuremath{\mathcal{B}}}\;A \;\AgdaDatatype{\ensuremath{\rVert}} \; \; \; \AgdaDatatype{\ensuremath{\lVert}}\; ((x : A) \rightarrow \AgdaDatatype{\ensuremath{\mathcal{B}}}\;(U\;x)) \;\AgdaDatatype{\ensuremath{\rVert}}
  }{
    \AgdaDatatype{\ensuremath{\lVert}}\; \AgdaDatatype{\ensuremath{\mathcal{B}}}\;((x : A) \rightarrow U\;x) \;\AgdaDatatype{\ensuremath{\rVert}}
  }
\end{equation*}
However our \emph{desired} closure proof is the following:
\begin{equation*}
  \frac{
    \AgdaDatatype{\ensuremath{\lVert}}\; \AgdaDatatype{\ensuremath{\mathcal{B}}}\;A \;\AgdaDatatype{\ensuremath{\rVert}} \; \; \; ((x : A) \rightarrow \AgdaDatatype{\ensuremath{\lVert}}\; \AgdaDatatype{\ensuremath{\mathcal{B}}}\;(U\;x) \;\AgdaDatatype{\ensuremath{\rVert}})
  }{
    \AgdaDatatype{\ensuremath{\lVert}}\; \AgdaDatatype{\ensuremath{\mathcal{B}}}\;((x : A) \rightarrow U\;x) \;\AgdaDatatype{\ensuremath{\rVert}}
  }
\end{equation*}

The solution would be to find a function of the following type:
\begin{equation*}
  ((x : A) \rightarrow \AgdaDatatype{\ensuremath{\lVert}}\; \AgdaDatatype{\ensuremath{\mathcal{B}}}\;(U\;x) \;\AgdaDatatype{\ensuremath{\rVert}}) \rightarrow
  \AgdaDatatype{\ensuremath{\lVert}}\; (x : A) \rightarrow \AgdaDatatype{\ensuremath{\mathcal{B}}}\;(U\;x) \;\AgdaDatatype{\ensuremath{\rVert}}
\end{equation*}
Interestingly, this is similar to an encoding of the axiom of choice.
\begin{definition}[Axiom of Choice]\label{axiom-of-choice}
  In HoTT, the axiom of choice is commonly defined as follows \cite[lemma
  3.8.2]{hottbook}.
  For any set \(A\), and a type family \(U\) which is a set at all the points
  of \(A\), the following function exists:
  \begin{equation*}
    \left( (x : A) \rightarrow  \AgdaDatatype{\ensuremath{\lVert}}\; U(x) \;\AgdaDatatype{\ensuremath{\rVert}} \right) \rightarrow \AgdaDatatype{\ensuremath{\lVert}}\; (x : A) \rightarrow U(x) \;\AgdaDatatype{\ensuremath{\rVert}}
  \end{equation*}
\end{definition}
Luckily, choice does hold on finite types.
\begin{lemma}
  The axiom of choice holds for finite sets.
  \begin{equation*}
    \agdacal{C}\;A \rightarrow ((x : A) \rightarrow \AgdaDatatype{\ensuremath{\lVert}}\; U(x) \;\AgdaDatatype{\ensuremath{\rVert}}) \rightarrow \AgdaDatatype{\ensuremath{\lVert}}\; (x : A) \rightarrow U(x) \;\AgdaDatatype{\ensuremath{\rVert}}
  \end{equation*}
\end{lemma}
\begin{proof}
  Let \(A\) be a cardinally finite type, \(U\) be a type family on \(A\), and
  \(f\) be a dependent function of type \(\Pi(x : A) , \AgdaDatatype{\ensuremath{\lVert}}\; U(x) \;\AgdaDatatype{\ensuremath{\rVert}}\).

  First, since our goal is itself propositionally truncated, we have access to
  values under truncations: put another way, in the context of proving our goal,
  we can rely on the fact that \(A\) is manifestly Bishop finite.
  Using the same technique as we did in \Cref{split-enum-pi}, we can switch
  from working with dependent functions from \(A\) to \(n\)-tuples, where \(n\)
  is the cardinality of \(A\).
  This changes our goal to the following:
  \begin{equation}
    \AgdaFunction{Tuple}\;n\;(\AgdaDatatype{\ensuremath{\lVert\_\rVert}}\;\AgdaFunction{\ensuremath{\circ}}\; U) \rightarrow \AgdaDatatype{\ensuremath{\lVert}}\; \AgdaFunction{Tuple}\;n\;U\;\AgdaDatatype{\ensuremath{\rVert}}
  \end{equation}
  Since \(\AgdaDatatype{\ensuremath{\lVert\_ \rVert}}\) is closed under finite products, this function
  exists (in fact, using the fact that \(\AgdaDatatype{\ensuremath{\lVert\_ \rVert}}\) forms a monad, we
  can recognise this function as \verb+sequenceA+ from the \verb+Traversable+
  class in Haskell).
\end{proof}
This lemma is a well-known folklore theorem.

With the closure properties proven, the following theorem is complete:
\begin{theorem}\label{kuratowski-topos}
  Decidable Kuratowski finite sets form a \(\Pi\)-pretopos.
\end{theorem}
