\section{Topos}\label{topos}
In this subsection we will examine the categorical interpretation of finite sets.
In particular, we will prove that discrete Kuratowski finite types form a
\(\Pi\)-pretopos.
A lot of the work for this proof has been done already: we have already proven
that discrete Kuratowski finiteness is equivalent to cardinal finiteness
(\Cref{cardinal-kuratowski}), meaning that we can work with the latter
definition which is much simpler to prove things about.

There are two reasons we're interested in the categorical and topos-theoretic
interpretation of finite sets: first, it's an important theoretical grounding
for finite sets, which allows us to understand them in the context of other
set-like constructions.
Secondly, and more practically, the language of a topos is (or in our case the
\(\Pi\)-pretopos) is a common standard framework for doing mathematics
generally.
This makes it a good basis for an API for building QuickCheck-like generators,
for example.
\subsection{Categories in HoTT}
At first glance, HoTT seems like a perfect setting for category theory: the
univalence axiom identifies isomorphisms with equality, a useful tool for
category theory missing from MLTT.
While this initial impression is broadly true, the construction of categories in
HoTT is unfortunately quite complex and involved.

Much of this subsection is simply a summary of parts of \citet[section
9]{hottbook}.
The formal proofs we provide are part translation of those proofs in that
section, part from \citep{iversenFredefoxCat2018}
\citep{huProofrelevantCategoryTheory2020}, and part our own.

First, we need to think about the type of objects and arrows.
We cannot, unfortunately, leave them unrestricted: because of the potential for
higher homotopy in HoTT types, we have
to restrict the type of arrows to just the sets.
This notion: that of a category with all the usual laws such that arrows are a
set, is called a \emph{precategory}.
\ExecuteMetaData[agda/Categories.tex]{precategory}
We will use long arrows to refer to morphisms within a category:
\ExecuteMetaData[agda/Categories.tex]{morph-arrow}

From here, we can define a notion of isomorphisms.
\ExecuteMetaData[agda/Categories.tex]{isomorphism}
It's a condition on this type which separates the precategories from the
categories: if it satisfies a form of univalence, it the precategory is a full
category.
\ExecuteMetaData[agda/Categories.tex]{cat-univalence}

\subsection{The Category of Sets}
Next we'll look at how to construct the category of sets (in the HoTT sense).
Much of this work comes directly from \citet{rijkeSetsHomotopyType2015} and
\citet[section 10]{hottbook} (the latter of which is in fact an updated and
slightly less detailed version of the former).
In particular, our treatment (and definition) of categories and topoi comes
directly from those works.
We have provided in the formalisation a proof that sets in CuTT form a
\(\Pi\)-pretopos: this proof (in HoTT) is in fact the main result of
\citet{rijkeSetsHomotopyType2015}; our contribution is simply the formalisation.

The objects are represented by a \(\Sigma\):
\ExecuteMetaData[agda/Snippets/Category.tex]{hset}


This will be quite similar to our objects for finite sets.

Since sets in HoTT don't form a topos, there are quite a few smaller lemmas we
need to prove to get as close as we can (a \(\Pi W\)-pretopos): we won't include
them here, other than the closure proofs in the following subsection.
\subsection{Closure}
The two most involved proofs for showing that discrete Kuratowski sets form a
\(\Pi\)-pretopos are those proofs that show closure under \(\Pi\) and
\(\Sigma\).
We will describe them here.

In~\citet[Theorem 4.21]{fruminFiniteSetsHomotopy2018}, Kuratowski finite types
are proven to be closed under surjections, products, and sums.
Here we prove closure under products and sums, but also functions, \(\Sigma\),
and \(\Pi\) (and furthermore our closure proofs are given on all of the
finiteness predicates that they apply to).

\paragraph{Closure of the Ordered Predicates}
First, we will show that split enumerability (and, by extension, manifest
enumerability) are closed under \(\Pi\) and \(\Sigma\).
This is the first stepping stone on our way to prove that cardinal finiteness is
closed under the same.

Practically speaking, these proofs also open up a wide number of other closure
proofs to us.
By proving that dependent products and sums are finite, we get the non-dependent
cases for free.

\begin{lemma}\label{split-enum-sigma}
  Split enumerability is closed under \(\Sigma\).
  \ExecuteMetaData[agda/Cardinality/Finite/SplitEnumerable.tex]{split-enum-sigma}
\end{lemma}
\begin{proof}
  Our task is to construct the two components of the output pair: the support
  list, and the cover proof.
  We'll start with the support list: this is constructed by taking the Cartesian
  product of the input support lists.
  
    \ExecuteMetaData[agda/Cardinality/Finite/SplitEnumerable.tex]{sup-sigma}
  
  We use do notation here because we're working the list monad: this applies the
  latter function (\(ys\)) to every element of the list \(xs\), and concatenates
  the results.

  To show that this does indeed cover every element of the target type is a
  little intricate, but not necessarily difficult.
\end{proof}

Next we'll look at closure under \(\Pi\).
In MLTT, this is of course not provable: since all of the finiteness predicates
we have seen so far imply decidable equality, and since we don't have any kind
of decidable equality on functions in MLTT, we know that we won't be able to
show that any kind of function is finite; even one like \(\AgdaDatatype{Bool}
\rightarrow \AgdaDatatype{Bool}\).

CuTT is not so restricted.
Since we have things like function extensionality and transport, we can indeed
prove the finiteness of function types.
Our proof here makes use directly of the univalence axiom, and makes use
furthermore of all the previous closure proofs.
\begin{theorem}\label{split-enum-pi}
  Split enumerability is closed under dependent functions
  (\(\Pi\)-types).
  
    \ExecuteMetaData[agda/Cardinality/Finite/ManifestBishop.tex]{pi-clos}
  
\end{theorem}
\begin{proof}
  Let \(A\) be a split enumerable type, and \(U\) be a type family from \(A\),
  which is split enumerable over all points of \(A\).

  As \(A\) is split enumerable, we know that it is also manifestly Bishop finite
  (\Cref{split-enum-to-manifest-bishop}), and consequently we know \(A
  \simeq \AgdaDatatype{Fin}\;n\), for some \(n\) (\Cref{bishop-equiv}).
  We can therefore replace all occurrences of \(A\) with \(\AgdaDatatype{Fin}\;n\),
  changing our goal to:
  \begin{equation*}
    \frac{
      \AgdaDatatype{\ensuremath{\mathcal{E}!}}\;(\AgdaDatatype{Fin}\;n) \; \; \; \left((x : \AgdaDatatype{Fin}\;n) \rightarrow \AgdaDatatype{\ensuremath{\mathcal{E}!}}\;\left( U\;x \right)\right)
    }{
      \AgdaDatatype{\ensuremath{\mathcal{E}!}}\left((x : \AgdaDatatype{Fin}\;n) \rightarrow U\;x\right)
    }
  \end{equation*}
  
  We then define the type of \(n\)-tuples over some type family.
  
    \ExecuteMetaData[agda/Data/Tuple/UniverseMonomorphic.tex]{tuple-def}
  
  We can show that this type is equivalent to functions (proven in our formalisation):
  
    \ExecuteMetaData[agda/Data/Tuple/UniverseMonomorphic.tex]{tuple-iso}
  
  And therefore we can simplify again our goal to the following:
  \begin{equation*}
    \frac{
      \AgdaDatatype{\ensuremath{\mathcal{E}!}}\;(\AgdaDatatype{Fin}\;n) \; \; \; ((x : \AgdaDatatype{Fin}\;n) \rightarrow \AgdaDatatype{\ensuremath{\mathcal{E}!}}\left( U\;x \right))
    }{
      \AgdaFunction{\ensuremath{\mathcal{E}!}}\;\left(\AgdaFunction{Tuple}\;n\;U\right)
    }
  \end{equation*}
  
  We can prove this goal by showing that \(\AgdaFunction{Tuple}\;n\;U\) is split
  enumerable: it is made up of finitely many products of points of \(U\), which
  are themselves split enumerable, and \agdatop, which is also split enumerable.
  \Cref{split-enum-sigma} shows us that the product of finitely many split
  enumerable types is itself split enumerable, proving our goal.
\end{proof}
\paragraph{Closure on Cardinal Finiteness}
Since we don't have a function of type \(\agdacal{C}\;A \rightarrow
\AgdaDatatype{\ensuremath{\mathcal{B}}}\;A\), closure proofs on \(\AgdaDatatype{\ensuremath{\mathcal{B}}}\) do not transfer over to
\agdacal{C} trivially (unlike with \(\AgdaDatatype{\ensuremath{\mathcal{E}!}}\) and \(\AgdaDatatype{\ensuremath{\mathcal{B}}}\)).
The cases for \agdabot, \agdatop, and \(\AgdaDatatype{Bool}\) are simple to adapt: we
can just propositionally truncate their Bishop finiteness proof.

Non-dependent operators like \AgdaFunction{\(\times\)},
\AgdaFunction{\(\uplus\)}, and \(\rightarrow\) are also relatively
straightforward: since \(\AgdaDatatype{\ensuremath{\lVert\_\rVert}}\) forms a
monad, we can apply \(n\)-ary functions to values inside it, combining them
together.

  \ExecuteMetaData[agda/Cardinality/Finite/ManifestBishop.tex]{times-clos-sig}

Into a truncated context:

  \ExecuteMetaData[agda/Cardinality/Finite/Cardinal.tex]{times-clos-impl}



Unfortunately, for the dependent type formers like \(\Sigma\) and \(\Pi\), the
same trick does not work.
We have closure proofs like:
\begin{equation*}
  \frac{
    \AgdaDatatype{\ensuremath{\mathcal{B}}}\;A \; \; \; ((x : A) \rightarrow \AgdaDatatype{\ensuremath{\mathcal{B}}}\;(U\;x))
  }{
    \AgdaDatatype{\ensuremath{\mathcal{B}}}\;((x : A) \rightarrow U\;x)
  }
\end{equation*}
If we apply the monadic truncation trick we can derive closure proofs like the
following:
\begin{equation*}
  \frac{
    \AgdaDatatype{\ensuremath{\lVert}}\; \AgdaDatatype{\ensuremath{\mathcal{B}}}\;A \;\AgdaDatatype{\ensuremath{\rVert}} \; \; \; \AgdaDatatype{\ensuremath{\lVert}}\; ((x : A) \rightarrow \AgdaDatatype{\ensuremath{\mathcal{B}}}\;(U\;x)) \;\AgdaDatatype{\ensuremath{\rVert}}
  }{
    \AgdaDatatype{\ensuremath{\lVert}}\; \AgdaDatatype{\ensuremath{\mathcal{B}}}\;((x : A) \rightarrow U\;x) \;\AgdaDatatype{\ensuremath{\rVert}}
  }
\end{equation*}
However our \emph{desired} closure proof is the following:
\begin{equation*}
  \frac{
    \AgdaDatatype{\ensuremath{\lVert}}\; \AgdaDatatype{\ensuremath{\mathcal{B}}}\;A \;\AgdaDatatype{\ensuremath{\rVert}} \; \; \; ((x : A) \rightarrow \AgdaDatatype{\ensuremath{\lVert}}\; \AgdaDatatype{\ensuremath{\mathcal{B}}}\;(U\;x) \;\AgdaDatatype{\ensuremath{\rVert}})
  }{
    \AgdaDatatype{\ensuremath{\lVert}}\; \AgdaDatatype{\ensuremath{\mathcal{B}}}\;((x : A) \rightarrow U\;x) \;\AgdaDatatype{\ensuremath{\rVert}}
  }
\end{equation*}
They don't match!

The solution would be to find a function of the following type:
\begin{equation*}
  ((x : A) \rightarrow \AgdaDatatype{\ensuremath{\lVert}}\; \AgdaDatatype{\ensuremath{\mathcal{B}}}\;(U\;x) \;\AgdaDatatype{\ensuremath{\rVert}}) \rightarrow
  \AgdaDatatype{\ensuremath{\lVert}}\; (x : A) \rightarrow \AgdaDatatype{\ensuremath{\mathcal{B}}}\;(U\;x) \;\AgdaDatatype{\ensuremath{\rVert}}
\end{equation*}
However we might be disheartened at realising that this is a required goal: the
above equation is \emph{extremely} similar to the axiom of choice!
\begin{definition}[Axiom of Choice] \label{axiom-of-choice}
  In HoTT, the axiom of choice is commonly defined as follows \cite[lemma
  3.8.2]{hottbook}.
  For any set \(A\), and a type family \(U\) which is a set at all the points
  of \(A\), the following function exists:
  \begin{equation*}
    \left( (x : A) \rightarrow  \AgdaDatatype{\ensuremath{\lVert}}\; U(x) \;\AgdaDatatype{\ensuremath{\rVert}} \right) \rightarrow \AgdaDatatype{\ensuremath{\lVert}}\; (x : A) \rightarrow U(x) \;\AgdaDatatype{\ensuremath{\rVert}}
  \end{equation*}
\end{definition}
Luckily the axiom of choice \emph{does} hold for cardinally finite types,
allowing us to prove the following:
\begin{lemma}
  The axiom of choice holds for finite sets.
  \begin{equation*}
    \agdacal{C}\;A \rightarrow ((x : A) \rightarrow \AgdaDatatype{\ensuremath{\lVert}}\; U(x) \;\AgdaDatatype{\ensuremath{\rVert}}) \rightarrow \AgdaDatatype{\ensuremath{\lVert}}\; (x : A) \rightarrow U(x) \;\AgdaDatatype{\ensuremath{\rVert}}
  \end{equation*}
\end{lemma}
\begin{proof}
  Let \(A\) be a cardinally finite type, \(U\) be a type family on \(A\), and
  \(f\) be a dependent function of type \(\Pi(x : A) , \AgdaDatatype{\ensuremath{\lVert}}\; U(x) \;\AgdaDatatype{\ensuremath{\rVert}}\).

  First, since our goal is itself propositionally truncated, we have access to
  values under truncations: put another way, in the context of proving our goal,
  we can rely on the fact that \(A\) is manifestly Bishop finite.
  Using the same technique as we did in \Cref{split-enum-pi}, we can switch
  from working with dependent functions from \(A\) to \(n\)-tuples, where \(n\)
  is the cardinality of \(A\).
  This changes our goal to the following:
  \begin{equation}
    \AgdaFunction{Tuple}\;n\;(\AgdaDatatype{\ensuremath{\lVert\_\rVert}}\;\AgdaFunction{\ensuremath{\circ}}\; U) \rightarrow \AgdaDatatype{\ensuremath{\lVert}}\; \AgdaFunction{Tuple}\;n\;U\;\AgdaDatatype{\ensuremath{\rVert}}
  \end{equation}
  Since \(\AgdaDatatype{\ensuremath{\lVert\_ \rVert}}\) is closed under finite products, this function
  exists (in fact, using the fact that \(\AgdaDatatype{\ensuremath{\lVert\_ \rVert}}\) forms a monad, we
  can recognise this function as \verb+sequenceA+ from the \verb+Traversable+
  class in Haskell).
\end{proof}
This lemma is a well-known folklore theorem.

This gets us all of the necessary closure proofs on \agdacal{C}, and as a result
we know the following theorem.
\begin{theorem}\label{kuratowski-topos}
  Decidable Kuratowski finite sets form a \(\Pi\)-pretopos.
\end{theorem}
\subsection{The Absence of the Subobject Classifier}
It's a little unsatisfying that our topos construction has so many caveats: we
have to prove a lot of small, uninteresting lemmas just to get to a
\(\Pi\)-pretopos, all because we can't prove the one or two larger, simple
lemmas which would show that sets form a topos. 
So what exactly are we missing?

Well, one of the characteristic features of topos theory is that there are a
wide variety of equivalent ways to show that something is a topos (a natural
consequence of their being a wide variety of things which qualify as toposes).
For the direction we have been going, though, the big missing feature is the
\emph{subobject classifier}.

A subobject in this context refers to a subset.
In set theory, we can often describe a subset of some set \(A\) with the
following notation:
\begin{equation*}
  \left\{ x \;\vert\; x \in A ; P(x) \right\}
\end{equation*}
This is the subset of elements in \(A\) which satisfy some predicate \(P\).

Type theoretically, the way to express the same would be
\(\AgdaDatatype{\(\Sigma\)}\;A\;P\): if we wanted to describe the subset of \Nat
smaller than 10 we would write
\mbox{\(\AgdaDatatype{\(\Sigma\)[}\;\AgdaBound{n}\;\AgdaDatatype{:}\;\Nat\;\AgdaDatatype{]}\;\AgdaBound{n}\;\AgdaFunction{\(<\)}\;\AgdaNumber{10}\)}.
In general, however, this type holds too many elements to properly classify the
subsets of the larger set: there may be more than one inhabitant of \(P\;x\) for
any given \(x\).
For \emph{propositions}, however, (i.e. where \(P\) is a proposition),
\AgdaDatatype{\(\Sigma\)} represents a perfectly valid encoding of subsets.

The subobject classifier is an object within the topos (which must be a
contraction) which classifies monomorphisms (injections).
We can actually show that the ``subset'' notion we just defined does in fact
classify monomorphisms in sets in HoTT (in fact directly through univalence),
but at this point we run into our one and only size problem in this paper.
The actual object corresponding to the subobject classifier is the following:

  \ExecuteMetaData[agda/Snippets/Topos.tex]{prop-univ}

The problem here, crucially, is that the universe level of this type is one
higher than the universe level of the types it bounds.
In other words, this is \emph{not} an object in our \(\Pi\)-pretopos of sets,
where the types are all of universe level 0.

Remember that the purpose of universe levels was to prevent Girard's paradox.
However, there is an axiom which removes universe levels to a certain extent
which does \emph{not} imply the paradox: propositional resizing.
\begin{definition}[Propositional Resizing]
  The axiom of propositional resizing states that the following two types, for
  any universe level \(u\), are equivalent:
  
    \ExecuteMetaData[agda/Snippets/Topos.tex]{prop-resize}
  
\end{definition}
If propositional resizing holds, then we \emph{can} in fact construct a
subobject classifier, for both sets and finite sets.

%%% Local Variables:
%%% mode: latex
%%% TeX-master: "../paper"
%%% End:
