\chapter{Programming and Proving in Cubical Agda}
Agda \cite{norellDependentlyTypedProgramming2008} is a dependently-typed, pure,
functional programming language and proof assistant.
In this thesis, we will use it to explore and prove things about finite and
countable types.
Along the way, we will learn about proofs in a dependent setting, functional
programming, and Homotopy Type Theory.

In this chapter we will introduce the language with some basic examples, and
explain a little about how to program and prove in Agda.
Some Haskell knowledge will help, as much of the syntax (any many concepts) are
similar, but it is possible to struggle through without it.
It is recommended to try out the code examples in your own editor, or to look at
them in the real Agda files in the source.
\todo{Set this up: organise code examples better?}
\section{Basic Functional Programming in Agda}
The basic unit of functionality in Agda is the \emph{type}.
Let's define a type: the type of booleans.
\begin{agdalisting*}
  \ExecuteMetaData[agda/Snippets/Bool.tex]{bool-def}
\end{agdalisting*}
There's a lot of syntax wrapped up in this small snippet.
In prose, it provides four basic pieces of information:
\begin{enumerate}
  \item We are defining a new \AgdaKeyword{data} type.
  \item Its name is \AgdaDatatype{Bool}.
  \item \AgdaDatatype{Bool} is a \AgdaFunction{\(\text{Type}_0\)} kind of thing.
  \item There are two ways to construct values of type \AgdaDatatype{Bool}:
    \AgdaInductiveConstructor{false} and \AgdaInductiveConstructor{true}.
\end{enumerate}
Let's explain each piece one by one.

The last point is the simplest: we have listed the ways to construct values of
type \AgdaDatatype{Bool}.
Two ways, in fact, \AgdaInductiveConstructor{true} and
\AgdaInductiveConstructor{false}, and they're called the constructors.
We can use these constructors in programs by (for instance) assigning them to
variables.
\begin{agdalisting*}
  \ExecuteMetaData[agda/Snippets/Bool.tex]{bool-val}
\end{agdalisting*}
Here we've declared a variable\footnotemark called \AgdaFunction{a-boolean} with
the type \AgdaDatatype{Bool}, and said it is equal to the value
\AgdaInductiveConstructor{true}.

\footnotetext{Note that although we use the term ``variable'', the value of the
  variable \AgdaFunction{a-boolean} can not change.
  We couldn't reassign it on the following line.}

Now back to the first point: we say that we're defining a new \AgdaKeyword{data}
type.
Using the ``\AgdaKeyword{data}'' keyword is just one of the many ways of
defining types: it basically means that we are going to define the type by
listing all of its constructors.
Another way to define types is with \AgdaKeyword{record}, which we'll see later,
and yet another way is to define a type by referring to other already-defined types.
Here, for instance, we can define the type \AgdaFunction{Boolean}:
\begin{agdalisting*}
  \ExecuteMetaData[agda/Snippets/Bool.tex]{boolean}
\end{agdalisting*}
This snippet says ``I am defining a new thing called \AgdaFunction{Boolean}, it
is a \AgdaFunction{\(\text{Type}_0\)}, and it is equal to \AgdaDatatype{Bool}''.
Of course this isn't a very interesting declaration: as the equals sign implies,
\AgdaFunction{Boolean} is the same as \AgdaFunction{Bool} (other than the
spelling).
We've basically defined a synonym for the old type.

The third point is the most interesting: we say that \AgdaDatatype{Bool} is a
``\AgdaFunction{\(\text{Type}_0\)}'' kind of thing.
What does this mean?

Well, we've seen that we can assign types to variables just as easily as we
might assign values to variables: this is what was happening in the
\AgdaFunction{Boolean} example.
In fact, in Agda, there is no real distinction between ``types'' and ``values'':
types like \AgdaDatatype{Bool} \emph{are} values, just as much as
\AgdaInductiveConstructor{true} or \AgdaInductiveConstructor{false}!
This means that our types must themselves have types: hence we say that
\AgdaFunction{Boolean} has type \AgdaFunction{\(\text{Type}_0\)}.

But why the subscript 0?
Well we know that types are values in Agda, and so they themselves have types.
We know that the type of \AgdaDatatype{Bool} is
\AgdaFunction{\(\text{Type}_0\)}.
But what's the type of \AgdaFunction{\(\text{Type}_0\)}?
It turns out that if we say:
\begin{agdalisting*}
  \(\AgdaFunction{\(\text{Type}_0\)} :  \AgdaFunction{\(\text{Type}_0\)}\)
\end{agdalisting*}
We actually introduce a paradox into the language: Girard's paradox
\cite{girardInterpretationFonctionelleElimination1972}.
This is the type-theoretic analogue of Russell's paradox, and, if present, it
would allow us to prove things that are not true.
So we disallow it.

Dependently-types programming languages have many different ways of resolving
the issue: Agda's approach is called \emph{universe polymorphism}.
Basically, we say that the type of \AgdaInductiveConstructor{true} is
\AgdaDatatype{Bool}, the type of \AgdaDatatype{Bool} is
\AgdaFunction{\(\text{Type}_0\)}, the type of \AgdaFunction{\(\text{Type}_0\)}
is \AgdaFunction{\(\text{Type}_1\)}, the type of \AgdaFunction{\(\text{Type}_1\)}
is \AgdaFunction{\(\text{Type}_2\)}, and so on.

To be honest, avoiding Girard's paradox is one of things that isn't done
especially well in dependently-typed languages: most approaches require quite a
bit of tedious busywork from the programmer, and it's quite rare that a
programmer would run into a genuine universe size issue that exposes a deep
logical impossibility (we will run into one of the few cases in this thesis).
Most of the time, managing universe levels amounts to bookkeeping.
For that reason, and also because the current system of universe polymorphism in
Agda is quite under flux and likely to be changed soon, we won't spend too much
time on the topic.
Every code example provided is as universe-polymorphic as possible, though.

\section{An Expression Evaluator}
\begin{agdalisting}
  \ExecuteMetaData[agda/Snippets/Bool.tex]{not-def}
\end{agdalisting}
Later we will builder a verified solver for the Countdown problem
\cite{huttonCountdownProblem2002} \todo{Find original reference for this}, for
now we will look at building a small part of the solution: an expression
evaluator.
Along the way, we will see how to implement some common functional programming
idioms in Agda (with some small improvements over Haskell when it comes to
syntax), and we will see how dependent types can give us a slightly nicer
interface than normal.

First, we'll need to define the natural numbers:
\begin{agdalisting}
  \ExecuteMetaData[agda/Snippets/Nat.tex]{nat-def}
\end{agdalisting}
In contrast to \AgdaDatatype{Bool}, we haven't simply listed the inhabitants of
this type, as that would somewhat bloat the page count of this thesis.
Instead, we list the \emph{ways} to construct elements of the type.
First, we say that you can create a natural number called
\AgdaInductiveConstructor{zero}.
Then, we say if you already have a natural number, than you can create its
successor, using \AgdaDatatype{suc}.
For now, we can think of inductive type definitions in Agda as being:
\begin{enumerate}
  \item A type name.
  \item A kind (i.e. \AgdaFunction{\(\text{Type}_0\)}).
  \item A list of constructors, which are functions whose return types are the
    type being defined.
\end{enumerate}
We will see later that this isn't the entire story, but it is a good enough
intuition for now.

Of course we can also define functions on the natural numbers, in much the same
way as we did on the booleans:
\begin{agdalisting} \label{sub-def}
  \ExecuteMetaData[agda/Snippets/Nat.tex]{sub-def}
\end{agdalisting}
Here we've defined subtraction.

We want to define a language of arithmetic expressions.
With countdown in mind, we'll only need to support four operators, which we can
define in a simple data type:
\begin{agdalisting}
  \ExecuteMetaData[agda/Snippets/Expr.tex]{op-def}
\end{agdalisting}

Next, we'll define the actual type of expressions.
\begin{agdalisting}
  \ExecuteMetaData[agda/Snippets/Expr.tex]{expr-def}
\end{agdalisting}
What we've defined here is actually a simple leafy binary tree.
The syntax for the second constructor is not so simple, however: it defines a
\emph{mixfix} operator.
Each underscore in \AgdaInductiveConstructor{\(\_\langle \_ \rangle\_\)}
represents a hole which expressions can be put into.
This allows us to use the constructor like so:
\begin{agdalisting}
  \ExecuteMetaData[agda/Snippets/Expr.tex]{example-expr}
\end{agdalisting}
\section{Safe Evaluation With Monads}
The next step is to write the evaluator for the type we have defined above.
There is a slight complication, however: some definable expressions don't have
defined evaluations.
Take the subtraction as defined on natural numbers, for instance
(Listing~\ref{sub-def}).
We really shouldn't be able to subtract a larger number from a smaller one, and
we would like to use the type system to prevent us from doing this.

The most common technique to solve this problem uses \AgdaDatatype{Maybe}:
\begin{agdalisting}
  \ExecuteMetaData[agda/Data/Maybe/Base.tex]{maybe-def}
\end{agdalisting}
This is the first \emph{parameterised} type we have seen: \AgdaDatatype{Maybe}
is a container with one or zero elements, but we haven't specified which type
can inhabit it.
It can actually be specialised to any type when we use it.
For our use, we will specialise it to \agdambb{N}:
\begin{agdalisting}
  \ExecuteMetaData[agda/Snippets/Expr.tex]{eval-ty}
\end{agdalisting}

We will define this function by pattern-matching.
The first case is relatively simple:
\begin{agdalisting}
  \ExecuteMetaData[agda/Snippets/Expr.tex]{lit-case}
\end{agdalisting}

The second two cases are slightly more complex: because we need to recursively
evaluate the subtrees of the expression in each case, we will need to check that
each of those returns a \AgdaInductiveConstructor{just} value before applying
the operator.
Luckily, we can use Agda's built-in idiom brackets to make this definition
a little cleaner: \todo{More explanation here}
\begin{agdalisting}
  \ExecuteMetaData[agda/Snippets/Expr.tex]{appl-cases}
\end{agdalisting}

Next, we will define subtraction.
As pointed out already, in this case we have to check to make sure that the
subtraction is valid.
Like Haskell, Agda supports do notation for this case:
\begin{agdalisting}
  \ExecuteMetaData[agda/Snippets/Expr.tex]{sub-case}
\end{agdalisting}

Finally, we will handle the division case.
Here, we want to pattern-match on the returned value of the recursive call.
Agda also provides syntax for that:
\begin{agdalisting}
  \ExecuteMetaData[agda/Snippets/Expr.tex]{div-case}
\end{agdalisting}
\todo{expand on monads applicatives etc in this section}
\section{Statically Proving the Evaluation is Safe}
Using this evaluator in practice can be a little annoying:
because it always returns a \AgdaDatatype{Maybe}, simple expressions which are
obviously valid still need to be checked at run-time.
\begin{agdalisting}
  \ExecuteMetaData[agda/Snippets/Expr.tex]{example-eval}
\end{agdalisting}
This is where Agda can add a little to the usual example for monads of an
expression evaluator: using dependent types, we can actually statically (and
automatically) prove that a given expression is valid, and evaluate it without
checking for \AgdaInductiveConstructor{nothing} safely.

First, we will need the following function:
\begin{agdalisting}
  \ExecuteMetaData[agda/Snippets/Expr.tex]{is-just}
\end{agdalisting}
This simple function can tell us if the result of evaluating an expression is
successful or not.
In other words, it can test if an expression is valid.

To use this statically, however, we will need to employ the following
\emph{dependent} function:
\begin{agdalisting}
  \ExecuteMetaData[agda/Snippets/Expr.tex]{tee}
\end{agdalisting}
This function turns our boolean values into types: \agdatop (tautology), or
\agdabot (impossibility).
These types are defined like so:
\begin{multicols}{2}
  \begin{agdalisting*}
    \ExecuteMetaData[agda/Snippets/Introduction.tex]{bot}
  \end{agdalisting*}  \columnbreak
  \begin{agdalisting*}
    \ExecuteMetaData[agda/Snippets/Introduction.tex]{top}
  \end{agdalisting*}
\end{multicols}
The first type here, \agdabot, has no constructors: there are no values which
inhabit the type \agdabot.
Logically speaking, it is the type of falsehoods.
It is quite useful in practice: any function of type \(A \rightarrow \agdabot\)
we know can never return, so we know that it must be impossible to call such a
function.
In other words, the type \(A\) must not have any values which inhabit it.
As such, we can use \agdabot to define a notion of ``not'' for types:
\begin{agdalisting}
  \ExecuteMetaData[agda/Snippets/Introduction.tex]{not}
\end{agdalisting}

The second type, \agdatop, is a \AgdaKeyword{record}.
Types defined using \AgdaKeyword{record} are much more like classes or structs
in imperative programming language: instead of listing the constructors, we list
the \emph{fields} of these types.

Of course, in this case, our type doesn't have any fields.
Perhaps a more instructive example of a record is the following:
\begin{agdalisting}
  \ExecuteMetaData[agda/Snippets/Expr.tex]{pair}
\end{agdalisting}
Here we've defined the type of \emph{pairs}.

Types defined with \AgdaKeyword{data} and types defined with
\AgdaKeyword{record} are in some sense duals of each other: to \emph{consume} a
\AgdaKeyword{data} type, we have to handle each of the constructors; to \emph{construct}
a \AgdaKeyword{record} type, we have to handle each of the fields.
Another way to say this same thing is that \AgdaKeyword{data} types are sum
types, and \AgdaKeyword{record} types are products.
What we have in \agdabot and \agdatop is the identity for sums and products,
respectively.

Now, to be completely clear, we could absolutely have defined \agdatop as a
\AgdaKeyword{data} type with one constructor:
\begin{agdalisting}
  \ExecuteMetaData[agda/Snippets/Expr.tex]{data-top}
\end{agdalisting}
We use the \AgdaKeyword{record} definition simply because it tends to work a
little better in terms of ergonomics: basically, to construct a
\AgdaKeyword{record} type automatically, Agda attempts to construct all of its
\emph{fields} one by one.
Since \agdatop has no fields, this is an easy task, and hence Agda will be able
to automatically construct a value of type \agdatop in many situations
(We can ask Agda to construct something for us automatically by supplying an
underscore in place of where the value should go).
Agda is more conservative about automatically constructing \AgdaKeyword{data}
types, so there are fewer situations where it will do it automatically.
\todo{expand on this?}

So, now that we have a way of turning booleans into their logical equivalents
\todo{express this better} we can define a type for proofs that a given
expression is valid:
\begin{agdalisting}
  \ExecuteMetaData[agda/Snippets/Expr.tex]{valid}
\end{agdalisting}
A value of type \(\AgdaFunction{Valid}\;e\), for some expression \(e\), is a
proof that \(e\) doesn't have (for example) any divisions by zero, or
arithmetic underflows.

With this we can define a function which uses the statically provided proof in
order to rule out certain cases in a pattern-match, thereby giving us a function
which statically evaluates expressions without using a \AgdaDatatype{Maybe}:
\todo{still have to explain here implicit arguments, ``with'', etc}
\begin{agdalisting}
  \ExecuteMetaData[agda/Snippets/Expr.tex]{static-eval}
\end{agdalisting}
Notice here that the \AgdaFunction{Valid} proof is provided automatically,
enabled by the fact that we defined \agdatop as a record.

And with that we can statically evaluate expressions like so:
\begin{agdalisting} \label{example-static-eval}
  \ExecuteMetaData[agda/Snippets/Expr.tex]{example-static-eval}
\end{agdalisting}
\section{Equalities}
We actually have encountered our first ``proof'' with dependent types: we have
proven that a given expression is valid or not.
Now we're going to look at another kind of proof: one that shows that an
expression is \emph{equal} to something.
To do so we'll first have to explore path types in Cubical Agda.
\begin{definition}[Path Types]
  A proof that two values are equal in Cubical Agda is represented by a
  \emph{path}.
  This path will be denoted with the symbol \AgdaFunction{\(\equiv\)}.
  In other words, a value of type \(x\;\AgdaFunction{\(\equiv\)}\;y\) is a proof
  that \(x\) equals \(y\).
\end{definition}

Equalities as paths is the first topic we have reached where Cubical Type Theory
begins to differ from traditional Martin-Löf Type Theory.
There, we would usually define the type of proofs of equality like so:
\begin{agdalisting}
  \ExecuteMetaData[agda/Snippets/Equality.tex]{equality-def}
\end{agdalisting}
This is an inductive \AgdaKeyword{data} type, with one constructor: the
constructor can only be used when the two parameters to the type are the same,
meaning a value of this type contains a proof that they are the same.
We can retrieve this proof by pattern-matching on that constructor.

This is actually a perfectly usable equality type in CuTT: although the
elimination rule is a little complex and we won't look into it just yet.
However we prefer to represent equalities in a slightly more primitive way, as
it turns out to be a little more flexible.
This is the \emph{path} representation.

When represented as a path, an equality between two values of type \(A\)
actually behaves more like a function from \AgdaDatatype{I} to \(A\).
\AgdaDatatype{I} here is the type of the interval: it ranges from
\AgdaInductiveConstructor{i0} to \AgdaInductiveConstructor{i1}.
So, as a function then, when the path \(x\;\AgdaFunction{\(\equiv\)}\;y\) is
applied to \AgdaInductiveConstructor{i0}, it returns \(x\), and when it is
applied to \AgdaInductiveConstructor{i1}, it returns \(y\).

Already we can manipulate paths in some interesting ways.
First, we can manipulate values in the interval: we can take the inverse of a
point in the interval, for instance.
It's worth thinking about what this ``inverse'' corresponds to in the equality:
we will name it in the next listing.
\begin{agdalisting}
  \ExecuteMetaData[agda/Snippets/Equality.tex]{sym-def}
\end{agdalisting}
We will see some more intricate ways to manipulate paths later on, but for now
the ``function from an interval'' intuition is enough to understand the basics.
\section{Some Proofs of Equality}
So now that we know something about the equality type, let's put it to some use.
We can construct equality proofs of things which are ``obviously equal'' with
the following function:
\begin{agdalisting}
  \ExecuteMetaData[agda/Snippets/Equality.tex]{refl-def}
\end{agdalisting}
With this we can prove that the output from Equation.~\ref{example-static-eval}
is 8:
\begin{agdalisting*}
  \ExecuteMetaData[agda/Snippets/Expr.tex]{example-static-proof}
\end{agdalisting*}

Of course, these proofs aren't very interesting.
Something a little more complex might be the following:
\begin{agdalisting}
  \ExecuteMetaData[agda/Data/Nat/Properties.tex]{plus-assoc}
\end{agdalisting}
Unfortunately we can't look at much more complex proofs without building up some
more machinery around path types: we can't currently compose paths, for
instance.
\section{Quotients}
We've seen that data types can be defined by listing their constructors, where
each constructor is just a function whose return type is the type being defined.
However, we've also seen that equalities are just functions from the interval.
If we combine these two notions, we can actually define a \emph{higher
  inductive} type.
\begin{definition}[Higher Inductive Type]
  A normal inductive type (like \AgdaDatatype{Bool}, or \agdambb{N}) is a type
  where its \emph{point} constructors are listed.
  A higher inductive type can have point constructors, but it can also have
  \emph{path} constructors: instead of adding new values to the type, these
  constructors add new equalities to the type.
\end{definition}

One of the nice aspects of CuTT is that higher inductive types arise naturally
from the ``function from an interval'' interpretation of path types.
Expand out the definition of \AgdaFunction{\(\equiv\)} in the following type,
for instance:
\begin{agdalisting}
  \ExecuteMetaData[agda/Snippets/Circle.tex]{circle-def}
\end{agdalisting}
We see that the \AgdaInductiveConstructor{loop} constructor, though odd looking,
still does represent a function whose return value is
\AgdaDatatype{\(\text{S}^1\)}.

Just with regards to this \AgdaDatatype{\(\text{S}^1\)} type: it's actually the
HoTT representation of the \emph{circle}.
We won't examine its more interesting properties all that much: however it is a
good example of the simplest type with complex homotopy, so we will use it to
demonstrate several HoTT principles.

A different HIT that we \emph{will} examine in depth, however, is the following:
\begin{definition}[Set Quotient]
  In CuTT we can define the type of sets quotiented by a relation \(R\) as
  follows:
  \begin{agdalisting*}
    \ExecuteMetaData[agda/Snippets/Quotient.tex]{quot-def}
  \end{agdalisting*}
\end{definition}

So we have three constructors for our set quotient type: first, a constructor
that takes a value of the underlying type (\(A\)), and constructs a value in the
quotiented type.
Second, we see a \emph{path} constructor: this constructor does the actual
quotienting in the type.
It says that if there exists a relation \(R\) between two elements \(x\) and
\(y\) then there is also a path between the elements
\(\AgdaInductiveConstructor{\([\)}\;x\;\AgdaInductiveConstructor{\(]\)}\)
and
\(\AgdaInductiveConstructor{\([\)}\;y\;\AgdaInductiveConstructor{\(]\)}\).

The third constructor is a little bit beyond what we can explain at this point:
it \emph{truncates} the higher homotopy out of the type.
Effectively, it makes the type slightly more ``well-behaved'' with regards to
HoTT internals: without this constructor, the type would have a lot of
interesting structure which is far too interesting for us at this point.
\todo{introduce explanation now or later?}

We are going to define a type similar to the one above, but for expressions.
It will quotient out common arithmetic identities:
\begin{agdalisting}
  \ExecuteMetaData[agda/Snippets/Expr.tex]{quot-expr}
\end{agdalisting}
To evaluate this expression, we have to actually prove that the evaluation
function respects the given equality.


%%% Local Variables:
%%% mode: latex
%%% TeX-master: "../paper"
%%% End: