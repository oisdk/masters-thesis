\section{Countably Infinite Types} \label{infinite}
We have now built up a substantial amount of theory relating to finite types.
In this section, we will look at the \emph{countable} types: we will see that
there is a parallel kind of classification of predicates to the finiteness
predicates, with some notable differences.
\subsection{Countability}
For our first countability predicate, we will mirror split enumerability:
\begin{definition}[Split Countability]
  A type is ``split countable'' if there is a \emph{stream} which contains all
  of its elements.
  \begin{agdalisting*}
    \ExecuteMetaData[agda/Cardinality/Infinite/Split.tex]{split-count}
  \end{agdalisting*}
\end{definition}
The similarity to split enumerability should be clear: the only difference
between the two definitions, in fact, is the type of the container.

For countability we use \emph{streams}: these are basically infinite lists.
To a Haskeller, normal lists themselves often fulfil this purpose, but in a
total language like Agda, we need a totally different type.
Lists, as an inductive type, are not permitted to be infinitely large.
\begin{definition}[Streams]
  In Agda the type of streams can be given as a container:
  \begin{agdalisting*}
    \ExecuteMetaData[agda/Container/Stream.tex]{stream-def}
  \end{agdalisting*}
  Although this definition is so simple it is more common to define it without
  reference to the usual container machinery:
  \begin{agdalisting*}
    \ExecuteMetaData[agda/Codata/Stream.tex]{stream-def}
  \end{agdalisting*}
  By inlining the definition of the container primitives it's clear that the two
  types are isomorphic, but by defining streams in these terms we're able to
  use things like the membership function on containers.
\end{definition}

In the previous sections we saw different flavours of finiteness which were
really just different flavours of relations to \AgdaFunction{Fin}.
Unsurprisingly, given the definition of streams, we will see in this subsection
that different flavours of countability are really just different flavours of
relations to \Nat.
Case in point: our definition of split enumerability is definitionally equal to
a split surjection from \Nat.
\begin{agdalisting*}
  \ExecuteMetaData[agda/Cardinality/Infinite/Split.tex]{split-surj}
\end{agdalisting*}
From this we can derive decidable equality, just like we could with split
enumerability. \todo{Explain better?}

We also have equivalents to manifest enumerability, or even cardinal finiteness,
in the countable setting: they are less interesting than split countability,
however.
\subsection{Closure}
The closure proofs are where countability begins to differ from split
enumerability.
We will see one closure proof that stays the same, one that is absent, and one
that is additional.
\paragraph{Instances for Finite Types}
Before we move on to the proper closure proofs, it is worth pointing out that
all (non-empty) finite types are also countable.
Split countability, like split enumerability, does not disallow duplicates: this
means that for any non-empty type we can simply repeat one of its elements
infinitely to produce a countability proof.
\paragraph{Closure Under \(\Sigma\)}
The proof that split enumerability was closed under \AgdaDatatype{\(\Sigma\)}
was quite straightforward:
we were able to use the ``normal'' pattern of taking the Cartesian product of
two lists in order to generate the finite support list for
\AgdaDatatype{\(\Sigma\)}.
Unfortunately this doesn't work for infinite types: the reason for which can be
seen in Figure~\ref{pairings}.

% \todo{Explain this notion of exploration better:
%   \begin{itemize}
%     \item Talk about how we have to figure out a surjection from the natural
%       numbers.
%     \item This is like trying to figure out a pattern which hits every number.
%     \item Or like a way to map some type to the numbers.
%     \item When doing this in some kind of closure-like proof, we can pretend
%       that the input types (i.e. \(A\) and \(B\) in \(A \times B\)) are simply
%       \(\mathbb{N}\).
%   \end{itemize}
% }

\begin{figure*}
  \centering
  \tikzcdset{
  cramped,
  every matrix/.append style={nodes={font=\scriptsize}},
  row    sep/normal=1em,
  column sep/normal=1em,
  }
  \begin{subfigure}[b]{.49\textwidth}
    \centering
    \begin{tikzcd}
      (1,e) \ar[rdddd, out=-45, in=135] & (2,e) \ar[rdddd, out=-45, in=135] & (3,e) \ar[rdddd, out=-45, in=135] & (4,e) \ar[rdddd, out=-45, in=135] & (5,e)        \\
      (1,d) \ar[u]     & (2,d) \ar[u]    & (3,d) \ar[u]    & (4,d) \ar[u]    & (5,d) \ar[u] \\
      (1,c) \ar[u]     & (2,c) \ar[u]    & (3,c) \ar[u]    & (4,c) \ar[u]    & (5,c) \ar[u] \\
      (1,b) \ar[u]     & (2,b) \ar[u]    & (3,b) \ar[u]    & (4,b) \ar[u]    & (5,b) \ar[u] \\
      (1,a) \ar[u]     & (2,a) \ar[u]    & (3,a) \ar[u]    & (4,a) \ar[u]    & (5,a) \ar[u]
    \end{tikzcd}
    \caption{Depth-First}
    \label{depth-first}
  \end{subfigure} \hfill
  \begin{subfigure}[b]{.49\textwidth}
    \centering
    \begin{tikzcd}
      (1,e) \ar[dr] & (2,e) \ar[dr]  & (3,e) \ar[dr]    & (4,e) \ar[dr] & (5,e) \\
      (1,d) \ar[dr] & (2,d) \ar[dr]  & (3,d) \ar[dr]    & (4,d) \ar[dr] & (5,d) \ar[u] \\
      (1,c) \ar[dr] & (2,c) \ar[dr]  & (3,c) \ar[dr]    & (4,c) \ar[dr] & (5,c) \ar[uul] \\
      (1,b) \ar[dr] & (2,b) \ar[dr]  & (3,b) \ar[dr]    & (4,b) \ar[dr] & (5,b) \ar[uuull, out=130, in=-50] \\
      (1,a) \ar[u]  & (2,a) \ar[uul, out=130, in=-50] & (3,a) \ar[uuull, out=130, in=-50] & (4,a) \ar[uuuulll, out=130, in=-50] & (5,a) \ar[uuuulll, out=130, in=-50]
    \end{tikzcd}
    \caption{Breadth-First}
    \label{breadth-first}
  \end{subfigure}
  \caption{Two possible products for the sets \(\left[ 1 \dots 5 \right]\) and
    \(\left[  a \dots e \right]\)}
  \label{pairings}
\end{figure*}

%%% Local Variables:
%%% mode: latex
%%% TeX-master: "../paper"
%%% End:

The depth-first pattern is what we used previously: this explores the first list
exhaustively before exploring anything other than the first element of the
second list.
This clearly won't work for streams, as it would mean that nothing other than
the first element of the second type could be found in the entire support
stream.

So instead we use the second pattern: breadth-first search.
The way we actually code this pattern up is a little complex: we treat the
search space as having several ``levels''.
Each item in each input list has a level (its position in that input list); the
output level for two items is the \emph{sum} of those levels.
In pseudo-set-builder notation: 
\begin{equation*}
  (\mathit{xs}\times\mathit{ys})_n =
    \left[ (\mathit{xs}_i , \mathit{ys}_j) \vert i \leftarrow \left[ 0 \ldots n \right] ; j \leftarrow \left[ 0 \ldots n \right] ; i + j = n  \right]
\end{equation*}
And then this is flattened to give the output support list.

One last detail of this function before we actually provide it: we use yet
another definition of lists in its implementation, instead of the normal lists,
as it is useful for termination proofs.
\begin{agdalisting*}
  \begin{multicols}{2} \centering
    \ExecuteMetaData[agda/Data/List/Kleene.tex]{plus-def} \columnbreak
    \ExecuteMetaData[agda/Data/List/Kleene.tex]{star-def}
  \end{multicols}
\end{agdalisting*}
This definition of lists interleaves the definition of non-empty lists with the
definition of possibly-empty lists.
This makes it much easier to switch between the two without conversion
functions.

Finally, we can provide the function which actually performs the breadth-first
search two streams.
\begin{agdalisting*}
  \ExecuteMetaData[agda/Cardinality/Infinite/Split.tex]{sigma-sup}
\end{agdalisting*}
The corresponding cover proof is relatively straightforward, so we don't include
it here.
\paragraph{Closure Under The Kleene Star}
One of the more useful types we can prove to be countably infinite is the
\emph{list}.
This proof is significantly more complex than the previous, however: we need to
find a pattern which eventually covers any given element of type
\AgdaDatatype{List}\;\(A\), given \AgdaFunction{\(\aleph\)!}\;\(A\).
Again we will tackle this pattern by first treating the exploration space as a
series of finite \emph{levels}: we can then concatenate all of these levels
together to a final exploration pattern.
The next trick is to figure out how to define those levels: we'll do it by
saying the lists of a given level \(n\) each must sum (after incrementing the
indices of each element) to \(n\).
That means that the 0th level consists only of the empty list, the 1st level
consists of the list \(\left[ 0 \right]\), the 2nd of \(\left[ 0 , 0
\right]\) and \(\left[ 1 \right]\), and so on.
Again, in pseudo set-builder notation:
\begin{equation*}
  \Nat\star_i \coloneqq \left[ \left[ \mathit{xs}_{j - 1} \mid j \in \mathit{js} \right] \mid \mathit{js} : \AgdaDatatype{List}\;\Nat ; \AgdaFunction{sum}\;\mathit{js} = i ; 0 \notin \mathit{js}  \right]
\end{equation*}
\todo{Include code?}
\paragraph{No Closure Under \(\Pi\)}
One closure proof that is certainly not possible is closure under \(\Pi\):
it is not the case that functions from countable types are themselves countable.
While this is a pretty basic fact in computer science, we include it here to
show how simple its proof in Agda can be:
\begin{agdalisting*}
  \ExecuteMetaData[agda/Cardinality/Infinite/Split.tex]{cantor-diag}
\end{agdalisting*}

%%% Local Variables:
%%% mode: latex
%%% TeX-master: "../paper"
%%% End:
