\chapter{Countably Infinite Types} \label{infinite}
We have now built up a substantial amount of theory relating to finite types.
In this chapter, we will look at the \emph{countable} types: we will see that
there is a parallel kind of classification of predicates to the finiteness
predicates, with some notable differences.
\section{Countability}
For our first countability predicate, we will mirror split enumerability:
\begin{definition}[Split Countability]
  A type is ``split countable'' if there is a \emph{stream} which contains all
  of its elements.
  \begin{agdalisting*}
    \ExecuteMetaData[agda/Cardinality/Infinite/Split.tex]{split-count}
  \end{agdalisting*}
\end{definition}
The similarity to split enumerability should be clear: the only difference
between the two definitions, in fact, is the type of the container.

For countability we use \emph{streams}: these are basically infinite lists.
To a Haskeller, normal lists themselves often fulfil this purpose, but in a
total language like Agda, we need a totally different type.
Lists, as an inductive type, are not permitted to be infinitely large.
\begin{definition}[Streams]
  In Agda the type of streams can be given as a container:
  \begin{agdalisting*}
    \ExecuteMetaData[agda/Container/Stream.tex]{stream-def}
  \end{agdalisting*}
  Although this definition is so simple it is more common to define it without
  reference to the usual container machinery:
  \begin{agdalisting*}
    \ExecuteMetaData[agda/Codata/Stream.tex]{stream-def}
  \end{agdalisting*}
  By inlining the definition of the container primitives it's clear that the two
  types are isomorphic, but by defining streams in these terms we're able to
  use things like the membership function on containers.
\end{definition}

In the previous sections we saw different flavours of finiteness which were
really just different flavours of relations to \AgdaFunction{Fin}.
Unsurprisingly, given the definition of streams, we will see in this section
that different flavours of countability are really just different flavours of
relations to \agdambb{N}.
Case in point: our definition of split enumerability is definitionally equal to
a split surjection from \agdambb{N}.
\begin{agdalisting*}
  \ExecuteMetaData[agda/Cardinality/Infinite/Split.tex]{split-surj}
\end{agdalisting*}
From this we can derive decidable equality, just like we could with split
enumerability. \todo{Explain better?}
\section{Closure}
The closure proofs are where countability begins to differ from split
enumerability.
We will see one closure proof that stays the same, one that is absent, and one
that is additional.
\subsection{Instances for Finite Types}
Before we move on to the proper closure proofs, it is worth pointing out that
all (non-empty) finite types are also countable.
Split countability, like split enumerability, does not disallow duplicates.
\todo{Include Proof}
\subsection{Closure Under \AgdaDatatype{\(\Sigma\)}}
The proof that split enumerability was closed under \AgdaDatatype{\(\Sigma\)}
was quite straightforward:
we were able to use the ``normal'' pattern of taking the Cartesian product of
two lists in order to generate the finite support list for
\AgdaDatatype{\(\Sigma\)}.
Unfortunately this doesn't work for infinite types: the reason for which can be
seen in Figure~\ref{pairings}.

\begin{figure*}
  \centering
  \tikzcdset{
  cramped,
  every matrix/.append style={nodes={font=\scriptsize}},
  row    sep/normal=1em,
  column sep/normal=1em,
  }
  \begin{subfigure}[b]{.49\textwidth}
    \centering
    \begin{tikzcd}
      (1,e) \ar[rdddd, out=-45, in=135] & (2,e) \ar[rdddd, out=-45, in=135] & (3,e) \ar[rdddd, out=-45, in=135] & (4,e) \ar[rdddd, out=-45, in=135] & (5,e)        \\
      (1,d) \ar[u]     & (2,d) \ar[u]    & (3,d) \ar[u]    & (4,d) \ar[u]    & (5,d) \ar[u] \\
      (1,c) \ar[u]     & (2,c) \ar[u]    & (3,c) \ar[u]    & (4,c) \ar[u]    & (5,c) \ar[u] \\
      (1,b) \ar[u]     & (2,b) \ar[u]    & (3,b) \ar[u]    & (4,b) \ar[u]    & (5,b) \ar[u] \\
      (1,a) \ar[u]     & (2,a) \ar[u]    & (3,a) \ar[u]    & (4,a) \ar[u]    & (5,a) \ar[u]
    \end{tikzcd}
    \caption{Depth-First}
    \label{depth-first}
  \end{subfigure} \hfill
  \begin{subfigure}[b]{.49\textwidth}
    \centering
    \begin{tikzcd}
      (1,e) \ar[dr] & (2,e) \ar[dr]  & (3,e) \ar[dr]    & (4,e) \ar[dr] & (5,e) \\
      (1,d) \ar[dr] & (2,d) \ar[dr]  & (3,d) \ar[dr]    & (4,d) \ar[dr] & (5,d) \ar[u] \\
      (1,c) \ar[dr] & (2,c) \ar[dr]  & (3,c) \ar[dr]    & (4,c) \ar[dr] & (5,c) \ar[uul] \\
      (1,b) \ar[dr] & (2,b) \ar[dr]  & (3,b) \ar[dr]    & (4,b) \ar[dr] & (5,b) \ar[uuull, out=130, in=-50] \\
      (1,a) \ar[u]  & (2,a) \ar[uul, out=130, in=-50] & (3,a) \ar[uuull, out=130, in=-50] & (4,a) \ar[uuuulll, out=130, in=-50] & (5,a) \ar[uuuulll, out=130, in=-50]
    \end{tikzcd}
    \caption{Breadth-First}
    \label{breadth-first}
  \end{subfigure}
  \caption{Two possible products for the sets \(\left[ 1 \dots 5 \right]\) and
    \(\left[  a \dots e \right]\)}
  \label{pairings}
\end{figure*}

%%% Local Variables:
%%% mode: latex
%%% TeX-master: "../paper"
%%% End:

The depth-first pattern is what we used previously: this explores the first list
exhaustively before exploring anything other than the first element of the
second list.
This clearly won't work for streams, as it would mean that nothing other than
the first element of the second type could be found in the entire support
stream.

So instead we use the second pattern: breadth-first search.
The way we actually code this pattern up is a little complex: we treat the
search space as having several ``levels''.
Each item in each input list has a level (its position in that input list); the
output level for two items is the \emph{sum} of those levels.
In pseudo-set-builder notation: 
\begin{equation*}
  (\mathit{xs}\times\mathit{ys})_n =
    \left[ (\mathit{xs}_i , \mathit{ys}_j) \vert i \leftarrow \left[ 0 \ldots n \right] ; j \leftarrow \left[ 0 \ldots n \right] ; i + j = n  \right]
\end{equation*}
And then this is flattened to give the output support list.

One last detail of this function before we actually provide it: we use yet
another definition of lists in its implementation, instead of the normal lists,
as it is useful for termination proofs.
\begin{agdalisting*}
  \begin{multicols}{2} \centering
    \ExecuteMetaData[agda/Data/List/Kleene.tex]{plus-def} \columnbreak
    \ExecuteMetaData[agda/Data/List/Kleene.tex]{star-def}
  \end{multicols}
\end{agdalisting*}
This definition of lists interleaves the definition of non-empty lists with the
definition of possibly-empty lists.
This makes it much easier to switch between the two without conversion
functions.

Finally, we can provide the function which actually performs the breadth-first
search two streams.
\begin{agdalisting*}
  \ExecuteMetaData[agda/Cardinality/Infinite/Split.tex]{sigma-sup}
\end{agdalisting*}
\subsection{Closure Under Kleene}
\subsection{No Closure Under \AgdaDatatype{\(\Pi\)}}


% \section{Closure}
% We know that countable infinity is not closed under the exponential (function
% arrow), so the only closure we need to prove is \(\Sigma\) to cover all of
% what's left.
% \begin{theorem} \label{split-countability-sigma}
%   Split countability is closed under \(\Sigma\).
% \end{theorem}
% We know that countable infinity is not closed under the exponential (function
% arrow), so the only closure we need to prove is \(\Sigma\) to cover all of
% what's left.
% To do this we have to take a slightly different approach to the functions we
% defined before.
% Figure~\ref{pairings} illustrates the reason why: previously, we used the
% depth-first product pairing for each support list.
% This diverges if the first list is infinite, never exploring anything other than
% the first element in the second list.
% Instead, we use here the cantor pairing function, which performs a breadth-first
% search of the pairings of both lists.

% Finally, while we have lost certain closure proofs by allowing for infinite
% types, we also \emph{gain} some: in particular the Kleene star.
% \begin{theorem}
%   Split countability is closed under Kleene star.
%   \begin{equation}
%     \aleph_0!(A) \rightarrow \aleph_0!(\mathbf{List}(A))
%   \end{equation}
% \end{theorem}
% Again, this proof requires a particular pattern to ensure productivity.
% The pattern here builds an intermediate stream \(\mathcal{KV}\) of non-empty
% lists from the input support stream \(\mathit{xs}\), which is subsequently
% flattened.
% \begin{equation}
%   \mathcal{KV}_i \coloneqq \left[ \left[ \mathit{xs}_{j - 1} \mid j \in \mathit{js} \right] \mid \mathit{js} \in \mathbf{List}(\mathbb{N}) ; \text{sum}(\mathit{js}) = i ; 0 \notin \mathit{js}  \right]
% \end{equation}

%%% Local Variables:
%%% mode: latex
%%% TeX-master: "../paper"
%%% End:
