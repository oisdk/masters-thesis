\section{Conclusion}
This paper has explored finiteness in the setting of Cubical Type Theory.
Hopefully, it will serve as a reasonable introduction to dependent type theory
for functional programmers which is a little different from most other
introductions, and which gives a taste of topics like HoTT which are on the
cutting edge of the field.
For those more experienced with dependent types, hopefully the paper makes some
argument as to the usefulness and importance of univalence in dependently typed
language, and demonstrates its power both theoretically and practically.

As for the theoretical contributions of the paper, we have provided a thorough
accounting of many of the ways to say something is ``finite'' in a
dependently typed programming language, and grounded that account in topos
theory.
In the future we hope to see a similar exploration of the countably infinite
types, and a connection of those predicates to the finite predicates.

We would also like to see more uses for finite types in dependently typed
programming: termination checking seems one obvious area where they could be
used more extensively.

Automated proof search libraries based on finiteness are common in the
dependently typed programming world.
In this paper, we have presented such a library which has the added power of
univalence: in the future we would like to see explorations of how to improve
the efficiency of the proof search, to make it more practical for larger
examples.
We would also like to see a similar library for the countably infinite types,
which can perform partial proof search.

%%% Local Variables:
%%% mode: latex
%%% TeX-master: "../paper"
%%% End: