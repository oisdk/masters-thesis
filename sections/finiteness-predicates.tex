\section{Finiteness Predicates}\label{finiteness-predicates}
\begin{figure*}
  \centering
  \begin{tikzcd}[cramped, row sep=small, column sep=1.8em]
    {} &
    {} \ar[ddddd, dash, start anchor={[yshift=4ex]}, end anchor={[yshift=-4ex]}] &
    \text{Non-discrete} &
    \ar[ddddd, dash, dashed, start anchor={[yshift=4ex]}, end anchor={[yshift=-4ex]}] &
    \text{Discrete} &
    {}
    \\ %%%%%%%%%%%%%%%%%%%%%%%%%%%%%%%%%%%%%%%%%%%%%%%%%%%%%%%%%%%%%%%%%%%%%%%%%
    \ar[rrrr, dash, start anchor={[xshift=-6ex]}, end anchor={[xshift=10ex]}] &
    &
    &
    &
    {}
    &
    {}
    \\ %%%%%%%%%%%%%%%%%%%%%%%%%%%%%%%%%%%%%%%%%%%%%%%%%%%%%%%%%%%%%%%%%%%%%%%%%
    \text{Ordered} &
    &
    \text{\parbox{2.6cm}{\centering Manifest Enumerable (\ref{manifest-enumerability})}}
    \ar[
      from=rr,
      bend left=15,
      crossing over,
      start anchor={[yshift=-2ex]west},
      end anchor={[yshift=-2ex]east}
      ]
    \ar[
      rr,
      bend left=15,
      "\text{Discrete}" description,
      start anchor={[yshift=2ex]east},
      end anchor={[yshift=2ex]west}
      ]
    \ar[ddd, crossing over]
    &
    &
    \text{\parbox{2.6cm}{\centering Split Enumerable (\ref{split-enumerability})}}
      \ar[d, xshift=-1ex, bend right=30]
    &
    \\ %%%%%%%%%%%%%%%%%%%%%%%%%%%%%%%%%%%%%%%%%%%%%%%%%%%%%%%%%%%%%%%%%%%%%%%%%
    &
    &
    &
    &
    \text{\parbox{2.6cm}{\centering Manifest Bishop (\ref{manifest-bishop-finiteness})}}
      \ar[u, xshift=1ex, bend right=30]
    &
    \\ %%%%%%%%%%%%%%%%%%%%%%%%%%%%%%%%%%%%%%%%%%%%%%%%%%%%%%%%%%%%%%%%%%%%%%%%%
    \ar[rrrr, dash, dashed, start anchor={[xshift=-6ex]}, end anchor={[xshift=10ex]}] &
    &
    &
    &
    {}
    &
    \\ %%%%%%%%%%%%%%%%%%%%%%%%%%%%%%%%%%%%%%%%%%%%%%%%%%%%%%%%%%%%%%%%%%%%%%%%%
    \text{Unordered} &
    {} &
    \text{\parbox{2.6cm}{\centering Kuratowski (\ref{kuratowski})}}
    \ar[rr, crossing over, bend left=15, "\text{Discrete}" description, start anchor=north east, end anchor=north west]
    \ar[from=rr, crossing over, bend left=15, start anchor=south west, end anchor=south east]
    &
    {}
    &
    \text{\parbox{2.2cm}{\centering Cardinal (\ref{cardinal-finiteness})}}
      \ar[from=uu, crossing over]
      \ar[
      uuu,
      crossing over,
      "\text{Ord}" {description, near end},
      end anchor=east,
      start anchor=east,
      in=-45, out=45
      ]
    &
    \\
    & & & & &
    \\
    & \ar[rrrr, "\text{More Restrictive}" description, end anchor={[xshift=-6ex]}]
    &
    &
    &
    &
    \ar[uuuuuu, "\text{More Informative}" {description, very near start}]
    {}
  \end{tikzcd}
  \caption{Classification of finiteness predicates according to whether they are
    discrete (imply decidable equality) and whether they imply a total order.}
  \label{finite-classification}
\end{figure*}%

%%% Local Variables:
%%% mode: latex
%%% TeX-master: "../paper"
%%% End:
This section will define and describe each of the five predicates in
\Cref{finite-classification}.
These predicates are classified along two axes: restrictiveness, and
informativeness.
Restrictiveness refers to how many types are able to satisfy the predicate:
there are fewer Cardinal finite (\Cref{cardinal-finiteness}) types than there
are Kuratowski (\Cref{kuratowski}).
Informativeness refers to how much information the predicate ``leaks''; a proof
of Manifest Bishop finiteness (\Cref{manifest-bishop-finiteness}), for instance,
reveals an ordering on the underlying set.
Each arrow in~\Cref{finite-classification} indicates an implication proof; for
instance, when discreteness is provided we can go from
Manifest enumerability~(\Cref{manifest-enumerability}) to split
enumerability~(\Cref{split-enumerability}).

\subsection{Split Enumerability}\label{split-enumerability}
The first finiteness predicate we will look at is split enumerability.
A split enumerable type is a type for which all of its elements can
be listed.
\begin{definition}[Split Enumerable Set]\label{split-enum-def}
  To say that some type \(A\) is split enumerable is to say that there is a list
  \(\mathit{support} : \AgdaDatatype{List}\;A\) such that any value \(x : A\) is in
  \(\mathit{support}\).
  \ExecuteMetaData[agda/Cardinality/Finite/SplitEnumerable/Container.tex]{split-enum-def}
  We call the first component of this pair the ``support'' list, and the second
  component the ``cover'' proof.
  An equivalent version of this predicate was called \verb+Listable+ in
  \cite{firsovDependentlyTypedProgramming2015}.
\end{definition}

The usual definition of lists is as follows:
\ExecuteMetaData[agda/Data/List/Base.tex]{list-def}
However, the predicate above uses a \emph{container}-based definition of lists.
\begin{definition}[Containers]\label{container-def}
  A container \cite{abbottContainersConstructingStrictly2005} is a pair
  \(S , P\) where \(S\) is a type, the elements of which are called
  \emph{shapes}, and \(P\) is a type family on \(S\), where
  the elements of \(P(s)\) are the \emph{positions}.
  We ``interpret'' a container into a functor like so:
  \ExecuteMetaData[agda/Container.tex]{container-interp}
\end{definition}

Containers are a generic way to encode functors; while they can be cumbersome to
use in practice, as a representation they are very flexible, and they will
greatly simplify some proofs later on.
Lists are encoded as a container like so:
\ExecuteMetaData[agda/Container/List.tex]{list-def}
This definition further relies on the $\AgdaDatatype{Fin}$ type:
\begin{paracol}{2}
\ExecuteMetaData[agda/Data/Fin/Base.tex]{fin-def-1}\switchcolumn%
\ExecuteMetaData[agda/Data/Fin/Base.tex]{fin-def-2}
\end{paracol}\noindent%
Here, \(\AgdaDatatype{⊎}\) is the disjoint union of two types,
$\AgdaDatatype{\ensuremath{\bot}}$ is the type with no inhabitants, and
$\AgdaDatatype{\ensuremath{\top}}$ is the type with one inhabitant.
As a result, the type $\AgdaDatatype{Fin}\;\AgdaBound{n}$ has precisely
$\AgdaBound{n}$ inhabitants.
It is a representation of a finite prefix of the natural numbers.
\begin{equation*}
  \AgdaDatatype{Fin}\;\AgdaBound{n} \simeq \left\{ m \; \vert \; 0 \leq m < n   \right\}
\end{equation*}


With this definition in hand, the container-based definition of lists 
says that ``Lists are a datatype whose shape
is given by the natural numbers, and which can be indexed by numbers smaller
than its shape''.
This representation is isomorphic to the standard one.

With containers we get a generic way to define ``membership'':\\
\twocolcode
{\ExecuteMetaData[agda/Snippets/Introduction.tex]{fiber}}
{\ExecuteMetaData[agda/Container/Membership.tex]{membership-def}}
Here we're using the homotopy-theory notion of a \AgdaDatatype{fiber} to define
membership: a fiber for some function \(f\) and some point \(y\) in its codomain
is a value \(x\) and a proof that \(f \; x \;\AgdaDatatype{≡}\;y\).
In the case of lists,
\(x\;\AgdaDatatype{\ensuremath{\in}}\;\mathit{xs}\) means ``there is an index
into \(\mathit{xs}\) such that the index points at an item equal to \(x\)''.

\paragraph{Split Surjections}
A common definition of finiteness requires a surjection from a finite prefix of
the natural numbers.
In HoTT, surjections (or, more precisely, \emph{split} surjections
\cite[definition 4.6.1]{hottbook}), are defined like so:

\twocolcode
{\ExecuteMetaData[agda/Function/Surjective/Base.tex]{split-surjective}}
{\ExecuteMetaData[agda/Function/Surjective/Base.tex]{split-surjection}}
As it turns out, our definition of finiteness here is precisely the same as a
surjection-based one.
\begin{lemma}\label{split-enum-is-split-surj}%
  A proof of split enumerability is equivalent to a split surjection from a
  finite prefix of the natural numbers.
  \ExecuteMetaData[agda/Cardinality/Finite/SplitEnumerable.tex]{is-split-inj-type}
\end{lemma}\noindent%
\textsc{Proof.}

\begin{minipage}[t]{.83\textwidth} \vspace{-1.25\baselineskip}
\ExecuteMetaData[agda/Cardinality/Finite/SplitEnumerable.tex]{is-split-inj}
\end{minipage}
\begin{minipage}[t]{.165\textwidth}
    Def.~\ref{split-enum-def} (\AgdaDatatype{\ensuremath{\mathcal{E}!}}) \\
    Eqn.~\ref{container-membership} (\AgdaDatatype{\ensuremath{\in}}) \\
    Eqn.~\ref{sp-surj-eqn}  \\
    Eqn.~\ref{list-def} (\(\AgdaDatatype{List}\)) \\
    Eqn.~\ref{container-interp}  \\
    Reassociation \\
    Eqn.~\ref{sp-surj-eqn}
\end{minipage}

In the above proof syntax the
\(\AgdaDatatype{\ensuremath{\equiv \langle{} \rangle{} }}\) connects lines which
are definitionally equal, i.e. they are ``obviously'' equal from the type
checker's perspective.
Only one line isn't a definitional equality: 
\ExecuteMetaData[agda/Data/Sigma/Properties.tex]{reassoc}
This means that we could have in fact written the whole proof as follows:
\ExecuteMetaData[agda/Cardinality/Finite/SplitEnumerable.tex]{split-enum-is-split-surj-short}
The fact that the two predicates are \emph{almost} definitionally equal is due
to the use of the container-based list definition.

To actually show that a type \(A\) is finite amounts to constructing a term of
type \(\AgdaDatatype{\ensuremath{\mathcal{E}!}}\;A\).
For simple types like \AgdaDatatype{Bool}, that is simple: it just amounts to
basically listing the constructors.
\begin{multicols}{2}
\ExecuteMetaData[agda/Cardinality/Finite/SplitEnumerable.tex]{bool-inst}
\end{multicols}

A slightly more complex example is the proof of finiteness of
\(\AgdaDatatype{Fin}\).
Since split enumerability is in fact the same as a split surjection from
\(\AgdaDatatype{Fin}\) (\Cref{split-enum-is-split-surj}):
to prove that $\AgdaDatatype{Fin}\;\AgdaBound{n}$ is finite we have to provide a
split surjection from $\AgdaDatatype{Fin}\;\AgdaBound{n}$ to itself.
Luckily, the identity function is a split surjection, so this holds.

\paragraph{Decidable Equality}
All split enumerable types are \emph{discrete}: they have decidable equality. 
\begin{paracol}{2} 
\ExecuteMetaData[agda/Relation/Nullary/Discrete/Base.tex]{discrete-def}
\switchcolumn%
\ExecuteMetaData[agda/Snippets/Dec.tex]{dec-def}
\end{paracol}
\begin{lemma}\label{split-enum-discrete}
  Split enumerability implies decidable equality.
\end{lemma}
\begin{proof}
  We will use the following definition of injections for this proof:
  \begin{paracol}{2}
    \ExecuteMetaData[agda/Function/Injective/Base.tex]{injective}\switchcolumn%
    \ExecuteMetaData[agda/Function/Injective/Base.tex]{injection}
  \end{paracol}

  Any type which injects into a discrete type is itself discrete:
  \ExecuteMetaData[agda/Function/Injective/Properties.tex]{inj-discrete}
  And any split surjection from \(A\) to \(B\) gives rise to an injection from
  \(B\) to \(A\):
  \ExecuteMetaData[agda/Function/Surjective/Properties.tex]{surj-to-inj}
  Yielding a simple proof that any type with a split surjection from a discrete
  type is itself discrete:
  \ExecuteMetaData[agda/Function/Surjective/Properties.tex]{discrete-surj}

  Since split enumerability is really just a split surjection from
  \AgdaDatatype{Fin}, and since we know that \AgdaDatatype{Fin} is discrete, the
  overall proof is:
  \ExecuteMetaData[agda/Cardinality/Finite/SplitEnumerable.tex]{split-is-discrete}
\end{proof}

\subsection{Manifest Bishop Finiteness}\label{manifest-bishop-finiteness}
As mentioned in the introduction, occasionally in constructive mathematics
proofs will contain information superfluous to the proposition in question.
For instance, in 
\begin{paracol}{2}
\ExecuteMetaData[agda/Cardinality/Finite/SplitEnumerable.tex]{bool-slop-1}\switchcolumn%
\ExecuteMetaData[agda/Cardinality/Finite/SplitEnumerable.tex]{bool-slop-2}
\end{paracol}\noindent%
There is an extra \AgdaInductiveConstructor{false} at the end of the support
list.
There is ``slop'' in the type of split enumerability: there are more distinct
values than there are \emph{usefully} distinct values.
To reconcile this, we will disallow duplicates in the support list.


A simple way to do this is to insist that every
\emph{membership proof} must be unique.
This would disallow a definition of \(\AgdaDatatype{\ensuremath{\mathcal{E}!}}\;
\AgdaDatatype{Bool}\) with
duplicates, as there are multiple values which inhabit the type
\(\AgdaInductiveConstructor{false}\;\AgdaFunction{\ensuremath{\in}}\;
\AgdaInductiveConstructor{\ensuremath{\left[ \text{false}, \text{true},
      \text{false} \right]}}\).
It also allows us to keep most of the split enumerability definition unchanged,
just adding a condition to the returned membership proof in the cover proof.

To specify that a value must exist uniquely in HoTT we can use the concept of a
\emph{contractible type}~\cite[definition 3.11.1]{hottbook}.
\ExecuteMetaData[agda/Snippets/Introduction.tex]{isContr}
A contractible type is one with the least possible amount of information: it
represents the tautologies.
All contractible types are isomorphic to \AgdaDatatype{\ensuremath{\top}}.

By saying that a proof of membership is contractible, we are saying that it
must be \emph{unique}.
\ExecuteMetaData[agda/Container/Membership.tex]{uniq-memb-def}
Now a proof of \(x\;\AgdaFunction{\ensuremath{\in}!}\;\mathit{xs}\) means that
\(x\) is not just in \(\mathit{xs}\), but it appears there \emph{only once}.

With this we can define manifest Bishop finiteness:
\begin{definition}[Manifest Bishop Finiteness]\label{bish-def}
  A type is manifest Bishop finite if there exists a list which contains each
  value in the type once.
  \ExecuteMetaData[agda/Cardinality/Finite/ManifestBishop/Container.tex]{bish-def}
  The only difference between manifest Bishop finiteness and split enumerability
  is the membership term: here we require unique membership
  (\AgdaFunction{\ensuremath{\in!}}), rather than simple membership
  (\AgdaFunction{\(\in\)}).
  An equivalent version of this predicate was called \verb+ListableNoDup+ in
  \cite{firsovDependentlyTypedProgramming2015}.
\end{definition}

We use the word ``manifest'' here to distinguish from another common
interpretation of Bishop finiteness, which we have called cardinal finiteness in
this paper: this version of the proof is ``manifest'' because we have a
concrete, non-truncated list of the elements in the proof.

\paragraph{The Relationship Between Manifest Bishop Finiteness and Split
  Enumerability}
While manifest Bishop finiteness might seem stronger than split enumerability,
it turns out this is not the case.
Both predicates imply the other.

Going from manifest Bishop finiteness is relatively straightforward:
to construct a proof of split enumerability from one of manifest Bishop
finiteness, it suffices to convert a proof of \(x\;\AgdaFunction{\ensuremath{\in!}}\;\mathit{xs}\) to
one of \(x\;\AgdaFunction{\ensuremath{\in}}\;\mathit{xs}\), for all \(x\) and \(\mathit{xs}\).
Since \AgdaFunction{\(\in!\)} is a contraction of \AgdaFunction{\(\in\)}, such a
conversion is simply the \AgdaField{fst} function.

Going the other direction takes significantly more work.
\begin{lemma}\label{split-enum-to-manifest-bishop}
  Any split enumerable set is manifest Bishop finite.
\end{lemma}
\begin{proof}
This lemma is proven in \cite{firsovDependentlyTypedProgramming2015}.
We will only sketch the proof here:
for a split enumerable type $A$, we know that it has decidable equality,
via~\Cref{split-enum-discrete}.
This allows us to write a function that removes duplicates from a list:
\begin{paracol}{2}
  \ExecuteMetaData[agda/Data/List/Membership.tex]{uniques}\switchcolumn%
  \ExecuteMetaData[agda/Data/List/Membership.tex]{remove}
\end{paracol}
Running this function on the support list of the split enumerability proof
yields a suitable support list for the manifest Bishop finiteness proof.
All that remains is to show that any element present in the original list is
also present in the created list
($\forall x \; \AgdaBound{xs}. \; x \in \AgdaBound{xs} \to x \in \AgdaFunction{uniques}\;\AgdaBound{xs}$),
and that every element in the created list is unique
($\forall x \; \AgdaBound{xs}. \; \AgdaFunction{isProp}\; (x \in \AgdaFunction{uniques}\;\AgdaBound{xs})$).
\end{proof}

We have now proved that every manifestly Bishop finite type is split enumerable,
and vice versa.
While the types are not \emph{equivalent} (there are more split enumerable
proofs than there are manifest Bishop finite proofs), they are of equal power.
\paragraph{From Manifest Bishop Finiteness to Equivalence}
We have seen that split enumerability was in fact a split-surjection in
disguise.
We will now see that manifest Bishop finiteness is in fact an \emph{equivalence}
in disguise.
We define equivalences as contractible maps \cite[definition 4.4.1]{hottbook}:
    \begin{paracol}{2}
      \ExecuteMetaData[agda/Snippets/Equivalence.tex]{is-equiv-def}\switchcolumn%
      \ExecuteMetaData[agda/Snippets/Equivalence.tex]{equiv-def}
      \synccounter{equation}%
    \end{paracol}
\begin{lemma}\label{bishop-equiv}
  Manifest bishop finiteness is equivalent to an equivalence to a finite prefix
  of the natural numbers.
  \ExecuteMetaData[agda/Cardinality/Finite/ManifestBishop.tex]{bishop-is-equiv-type}
\end{lemma}\noindent%
\textsc{Proof.}

\begin{minipage}[t]{.83\textwidth}\vspace{-1.25\baselineskip}
  \ExecuteMetaData[agda/Cardinality/Finite/ManifestBishop.tex]{bishop-is-equiv}
\end{minipage}
\begin{minipage}[t]{.16\textwidth}
  Def.~\ref{bish-def} (\AgdaDatatype{\ensuremath{\mathcal{B}}})     \\
  Eqn.~\ref{uniq-memb-def} (\AgdaDatatype{\ensuremath{\in!}})       \\
  Eqn.~\ref{container-membership} (\AgdaDatatype{\ensuremath{\in}}) \\
  Eqn.~\ref{equiv-def} \\
  Eqn.~\ref{list-def} ({\AgdaDatatype{List}}) \\
  Eqn.~\ref{container-interp}  \\
  Reassociation \\
  Eqn.~\ref{equiv-def}
\end{minipage}

This proof is almost identical to the proof for
\Cref{split-enum-is-split-surj}: it reveals that
enumeration-based finiteness predicates are simply another perspective on
relation-based ones.

\subsection{Cardinal Finiteness}\label{cardinal-finiteness}
While we have removed some of the unnecessary information from our finiteness
predicates, one piece still remains: the ordering.
A proof of manifest bishop finiteness includes a total order on the underlying
set.
For our purposes, this is too much information: it means that when constructing
the ``category of finite sets'' later on, instead of each type having one
canonical representative, it will have $n!$ where $n$ is the cardinality of
the type.\footnotemark

\footnotetext{
  We actually do get a category (a groupoid, even) from manifest Bishop
  finiteness \cite{yorgeyCombinatorialSpeciesLabelled2014}: it's the groupoid of
  finite sets equipped with a linear order, whose morphisms are order-preserving
  bijections.
  We do not explore this particular construction in any detail.
}

What is needed is a proof of finiteness that is a proposition.
\ExecuteMetaData[agda/Snippets/Introduction.tex]{isProp}
The mere propositions are one homotopy level higher than the contractible types
(\Cref{isContr}), the types for which all values are equal to some value.
They represent the types for which all values are equal.
Two propositions are {\agdatop} and {\agdabot}: in a classical setting, we can
say that these are the \emph{only} two propositions, that every proposition is
isomorphic to either {\agdatop} or {\agdabot}.
Constructively, we need to be a little more careful with our phrasing: we can
say there is no proposition which is not equal to either of these types
($\nexists (A : \AgdaDatatype{Type}).\; \AgdaFunction{isProp}\;A \wedge (A \neq \agdatop) \wedge (A \neq \agdabot)$).

You can also define propositions in terms of contractible types: propositions are
the types whose paths are contractible types.
Soon (\Cref{isSet}) we will see the next homotopy level, which are
defined in terms of the propositions.

Despite now knowing the precise property we want our finiteness predicate to
have, we're not much closer to achieving it.
To remedy the problem, we will use the following type:
\ExecuteMetaData[agda/Snippets/PropTrunc.tex]{prop-trunc-def}
This is a \emph{higher inductive type}.
Normal inductive types have \emph{point} constructors: constructors which
construct values of the type.
The first constructor here (\AgdaInductiveConstructor{\ensuremath{\lvert \_
    \rvert}}), or the constructor \AgdaInductiveConstructor{true} for
\AgdaDatatype{Bool}, are both ``point'' constructors.

What makes this type higher inductive is that it also has \emph{path}
constructors: constructors which add new equalities to the type. 
The \AgdaInductiveConstructor{squash} constructor here says that all elements of
\proptrunc{A}
are equal, regardless of what \(A\) is.
In this way it allows us to propositionally truncate types, turning
information-containing proofs into mere propositions.
Put another way, a proof of type \proptrunc{A}
is a proof that some \(A\) exists, without revealing \emph{which} \(A\).

To actually use values of this type we have the following eliminator:
\ExecuteMetaData[agda/HITs/PropositionalTruncation/Display.tex]{rec-prop-trunc}
This says that we can eliminate into any proposition: interestingly, this allows
us to define a monad instance for \AgdaDatatype{\(\lVert \_ \rVert\)}, meaning
we can use things like do-notation.

With this, we can define cardinal finiteness:
\begin{definition}[Cardinal Finiteness]\label{card-finite-def}
  A type \(A\) is cardinally finite if there exists a propositionally truncated
  proof that \(A\) is manifest Bishop finite.
  \ExecuteMetaData[agda/Cardinality/Finite/Cardinal.tex]{cardinal-def}
  This predicate is called Bishop finiteness in~\cite{fruminFiniteSetsHomotopy2018}.
\end{definition}

\paragraph{Deriving Uniquely-Determined Quantities}
At first glance, it might seem that we lose any useful properties we could
derive from \(\AgdaDatatype{\ensuremath{\mathcal{B}}}\).
Luckily, this is not the case: we will show here how to derive decidable
equality (\Cref{cardinal-finite-discrete}) and cardinality
(\Cref{card-finite-cardinality}) from under the
truncation.
Those two lemmas are proven in
\cite[Proposition 2.4.9 and 2.4.10]{yorgeyCombinatorialSpeciesLabelled2014}, in
much the same way as we have done here.
Our contribution for this subsection is simply the formalisation.

First we'll show that decidable equality carries over from manifest Bishop
finiteness.
Before we do, note that the fact that we can do this says something interesting
about propositional truncation: it has computational, or algorithmic, content.
That is in contrast to other ways to ``truncate'' types: \(\neg \neg P\), for
instance, is a way to provide a ``proof'' of \(P\) without revealing anything
about \(P\) in MLTT.
No matter how much we prove that a function from \(P\) doesn't care about which
\(P\) it got, though, we can never extract any kind of algorithm or computation
from \(\neg \neg P\).
\begin{lemma}\label{cardinal-finite-discrete}
  Any cardinal-finite set has decidable equality.
  \ExecuteMetaData[agda/Cardinality/Finite/Cardinal.tex]{card-discrete}
\end{lemma}
\begin{proof}
We already know that manifest Bishop finiteness implies decidable equality;
to apply that proof to cardinal finiteness we'll use the
eliminator in \Cref{elim-prop}.
Our task, in other words, is to prove the following:
\ExecuteMetaData[agda/Relation/Nullary/Discrete/Properties.tex]{is-prop-discrete}

To show that $\AgdaDatatype{Dec}\;(x\; \agdaequiv \; y)$ is a proposition it is
sufficient to show that $x \; \agdaequiv \; y$ is a proposition.
It turns out that there is a class of types for which all paths are
propositions: the \emph{sets}.
\ExecuteMetaData[agda/Snippets/Introduction.tex]{isSet}
This is the next homotopy level up from the propositions (\Cref{isProp}).
More importantly, there is an important theorem relating to sets which
\emph{also} relates to decidable equality: Hedberg's theorem
\cite{hedbergCoherenceTheoremMartinLof1998}.
This tells us that any type with decidable equality is a set.
\ExecuteMetaData[agda/Relation/Nullary/Discrete/Properties.tex]{discrete-isset}
And of course we know that \(A\) here has decidable equality.
\end{proof}

The next thing we can derive from underneath the truncation in cardinal
finiteness is a natural number representing the actual cardinality of the finite
type.
Of course {\Nat} isn't a proposition, so the eliminator in
equation~\ref{elim-prop} won't work for us here.
Instead we will use the following:
\ExecuteMetaData[agda/HITs/PropositionalTruncation/Display.tex]{rec-prop-trunc-set}
This says that we can eliminate into a set as long as the function we use
doesn't care about which value it's given: formally, \(f\) in this example has
to be ``coherently constant''~\cite{krausGeneralUniversalProperty2015}.

\begin{lemma}\label{card-finite-cardinality}
  Given a cardinally finite type, we can derive the type's cardinality, as well
  as a propositionally truncated proof of equivalence with \AgdaDatatype{Fin}s of
  the same cardinality.
  \ExecuteMetaData[agda/Cardinality/Finite/Cardinal.tex]{cardinality-is-unique}
\end{lemma}
\begin{proof}
  The high-level overview of our proof is as follows:
  
    \ExecuteMetaData[agda/Cardinality/Finite/Cardinal.tex]{cardinality-is-unique-impl}
  
  It is the composition of two operations: first, with
  \AgdaFunction{\ensuremath{\lVert \text{map} \rVert}}, we change the truncated
  proof of manifest bishop finiteness to a proof of equivalence with
  \AgdaDatatype{Fin}.

  Then we use the eliminator from \Cref{elim-prop-coh} with three
  parameters.
  The first simply proves that that the output is a set:
  
    \ExecuteMetaData[agda/Cardinality/Finite/Cardinal.tex]{card-isSet}
  
  The second is the function we apply to the truncated value:
  
    \ExecuteMetaData[agda/Cardinality/Finite/Cardinal.tex]{alg}
  
  And the third is a proof that that function is itself coherently constant:
  
    \ExecuteMetaData[agda/Cardinality/Finite/Cardinal.tex]{const-alg}
  

  The tricky part of the proof is \AgdaFunction{const-alg}: here we need to show
  that \AgdaFunction{alg} returns the same value no matter its input.
  That output is a pair, the first component of which is the cardinality, and
  the second the truncated equivalence proof.
  The truncated proofs in the output are trivially equal by the truncation, so
  our obligation now has been reduced to:
  \begin{equation*}
    \frac{(n :\;\AgdaDatatype{\ensuremath{\mathbb{N}}}) \; \; \; (p : \AgdaDatatype{Fin}\;n\;\AgdaFunction{\ensuremath{\simeq}}\;A) \; \; \;
      (m :\;\AgdaDatatype{\ensuremath{\mathbb{N}}}) \; \; \; (q : \AgdaDatatype{Fin}\;m\;\AgdaFunction{\ensuremath{\simeq}}\;A)
    }{
      n\;\agdaequiv\;m
    }
  \end{equation*}
  Given univalence we have \(\AgdaDatatype{Fin}\;n \;\AgdaFunction{\ensuremath{\equiv}}\; \AgdaDatatype{Fin}\;m\),
  and the rest of our task is to prove:
  \begin{equation*}
    \frac{{\AgdaDatatype{Fin}\;n}\;\agdaequiv\;{\AgdaDatatype{Fin}\;m}}{n\;\agdaequiv\;m}
  \end{equation*}

  This is a well-known puzzle in dependently typed programming, and one that
  has a surprisingly tricky and complex proof.
  We do not include it here, since it has already been explored elsewhere, but
  it is present in our formalisation.
\end{proof}

\paragraph{Going from Cardinal Finiteness to Manifest Bishop Finiteness}
We know of course that we can convert any proof of manifest Bishop finiteness to
a proof of Cardinal finiteness: it's just the truncation function
\AgdaInductiveConstructor{\(\lvert \_ \rvert\)}.
It's the other direction which presents a difficulty:
\begin{theorem}\label{cardinal-to-manifest-bishop}
  Any cardinal finite type with a (decidable) total order is Bishop finite.
\end{theorem}
\begin{proof}
  The following is a sketch of the proof, which is quite involved in the
  formalisation.

Our strategy will be to \emph{sort} the support list of the proof for Bishop
finiteness, and then prove that the sorting function is coherently constant,
thereby satisfying the eliminator in \Cref{elim-prop-coh}.
We have implemented insertion sort as $\AgdaFunction{sort}$, and proven that it
produces a sorted list which is a permutation of its input.
\begin{paracol}{2}
  \ExecuteMetaData[agda/Data/List/Sort.tex]{sort-sorts}
  \switchcolumn%
  \ExecuteMetaData[agda/Data/List/Sort.tex]{sort-perm}
\end{paracol}
\AgdaFunction{Sorted} is a predicate
enforcing that the given list is sorted, and
\AgdaFunction{\(\leftrightsquigarrow\)} is a permutation relation between two
lists.
The definition of permutations is from
\cite{danielssonBagEquivalenceProofRelevant2012}: two lists are permutations of
each other if their membership proofs are all equivalent.
\ExecuteMetaData[agda/Data/List/Relation/Binary/Permutation.tex]{perm-def}
This definition fits particularly well for two reasons: first, it is defined on
containers generically, which fits well with our finiteness predicates.
Secondly, it is extremely straightforward to show that the support lists of any
two proofs of manifest Bishop finiteness must be permutations of each other:
\ExecuteMetaData[agda/Cardinality/Finite/Cardinal.tex]{perm-bish}
The final piece of the puzzle is the following:
\ExecuteMetaData[agda/Data/List/Sort.tex]{sorted-perm-eq}
If two sorted lists are both permutations of each other they must be equal.
Connecting up all the pieces we get the following:

  \ExecuteMetaData[agda/Data/List/Sort.tex]{perm-invar}

Because we know that all support lists of \(\agdacal{B}\;A\) are permutations of
each other this is enough to prove that \AgdaFunction{sort} is coherently
constant, and therefore can eliminate from within a truncation.
The second component of the output pair (the cover proof) follows quite
naturally from the definition of permutations.
\end{proof}

\paragraph{Restrictiveness}
So far our explorations into finiteness predicates have pushed us in the
direction of ``less informative'': however, as mentioned in the introduction, we
can \emph{also} ask how \emph{restrictive} certain predicates are.
Since split enumerability and manifest Bishop finiteness imply each other we
know that there can be no type which satisfies one but not the other.
We also know that manifest Bishop finiteness implies cardinal finiteness, but we
do \emph{not} have a function in the other direction:
\begin{equation}\label{c-to-b}
  \agdacal{C}\;A\rightarrow\agdacal{B}\; A
\end{equation}
So the question arises naturally: is there a cardinally finite type which is
\emph{not} manifest Bishop finite?

The answer is no.
The proof of this fact is relatively short:
\ExecuteMetaData[agda/Cardinality/Finite/Cardinal.tex]{no-gap-card-bishop}
We can apply the function of type \(\agdacal{B}\;A\rightarrow\agdabot\)
(i.e. \(\AgdaFunction{\(\neg\)}\;\agdacal{B}\;A\)) to the value of type
\(\AgdaDatatype{\(\lVert\)}\;\agdacal{B}\;A\;\AgdaDatatype{\(\rVert\)}\) (i.e.
\(\agdacal{C}\;A\)) using \Cref{elim-prop}, since \agdabot\;is itself a
proposition.
This tells us that manifest bishop finiteness, cardinal finiteness, and split
enumerability all refer to the same class of types.

\subsection{Manifest Enumerability}\label{manifest-enumerability}
All of the finiteness predicates seen so far apply to the same types.
Other than obviously non-finite types (like {\Nat}), there are some exotic types
which do not conform to any of the previous predicates, but are in some sense
finite.
The \emph{circle} is such a type.
\ExecuteMetaData[agda/Snippets/Circle.tex]{circle-def}
The thing that this type has which precludes it from being, say, split
enumerable, is its \emph{higher homotopy structure}.

So far we have seen three levels of homotopy structure: the contractible types
(\Cref{isContr}), the propositions (\Cref{isProp}), and the sets
(\Cref{isSet}).
Notice the pattern: each new level is generated by saying its
paths are members of the previous level; if we apply that pattern again, we get
to the next homotopy level: the groupoids.
\ExecuteMetaData[agda/Snippets/Introduction.tex]{isGroupoid}
These types do not necessarily have unique identity proofs: there is more than
one value which can inhabit the type $x \; \agdaequiv \; y$.
The circle is one of the simplest examples of non-set groupoids: the constructor
\AgdaInductiveConstructor{loop} is the extra path in the type which isn't the
identity path.

Is the circle finite?
According to the finiteness predicates we have seen so far, it is not.
Recall that Hedberg's theorem says that every discrete type is a set, and
every finiteness predicate we've seen implies decidable
equality.

But this type is certainly finite in some other sense.
It has finitely many points, for instance.
To complete the ``restrictiveness'' axis in \Cref{finite-classification}, we
need a predicate which admits the circle.
Manifest enumerability is one such predicate.
\begin{definition}[Manifest Enumerability]
  Manifest enumerability is an enumeration predicate like Bishop finiteness or
  split enumerability with the only difference being a propositionally truncated
  membership proof.
  \ExecuteMetaData[agda/Cardinality/Finite/ManifestEnumerable/Container.tex]{manifest-enum-def}
\end{definition}

It might not be immediately clear why this definition of enumerability allows
the circle to conform while the others do not.
The crux of the issue was that the membership proofs of the previous definitions
didn't just prove that some element was in the support list, they revealed
\emph{where} it was in the support list.
This position information allows the derivation of decidable equality, and this
position is what is erased in manifest enumerability.

And indeed this means that the circle is manifestly enumerable.
\ExecuteMetaData[agda/Cardinality/Finite/ManifestEnumerable.tex]{circle-is-manifest-enum}

This uses a lemma, proven in the Cubical Agda library, that
\AgdaDatatype{S\(^1\)} is \emph{connected}:
\ExecuteMetaData[agda/Cardinality/Finite/ManifestEnumerable.tex]{s1-connected}

\paragraph{Surjections}
We already saw that split enumerability was the listed form of a split
surjection, however, in the presence of higher homotopies than sets, there is a
distinction between split surjections and surjections.
The following is the definition of a surjection which is not necessarily
split~\cite[definition 4.6.1]{hottbook}:
\begin{paracol}{2}
  \ExecuteMetaData[agda/Function/Surjective/Base.tex]{surjective}\switchcolumn%
  \ExecuteMetaData[agda/Function/Surjective/Base.tex]{surjection}
  \synccounter{equation}%
\end{paracol}

Much in the same way that split enumerability were split surjections, our new
predicate of manifest enumerability corresponds to the proper surjections.
\begin{lemma}
  Manifest enumerability is equivalent to a surjection from $\AgdaDatatype{Fin}$.
  \ExecuteMetaData[agda/Cardinality/Finite/ManifestEnumerable.tex]{manifest-enum-surj}
\end{lemma}

\paragraph{Relation To Split Enumerability}
It is trivially easy to construct a proof that any split enumerable type is
manifest enumerable: we simply truncate the membership proof.
Going the other way is more difficult, as we need to extract the membership
proof from under a truncation.
However, the function $\AgdaFunction{recompute}$ can do exactly this, when the
truncated type is decidable:
\ExecuteMetaData[agda/HITs/PropositionalTruncation/Properties.tex]{recompute}
And the membership proof in question is indeed decidable, because of the
presence of decidable equality.
\begin{lemma}\label{manifest-enum-to-split-enum}
  A manifestly enumerable type with decidable equality is split enumerable.
\end{lemma}

\subsection{Kuratowski Finiteness}\label{kuratowski}
We start with the definition of Kuratowski-finite subsets.
\ExecuteMetaData[agda/Algebra/Construct/Free/Semilattice/Definition.tex]{kuratowski-def}
The first two constructors are point constructors, giving ways to create
values of type \(\agdacal{K}\;A\).
They are also recognisable as the two constructors for finite lists, a type
which represents the free monoid.
The next two constructors add extra equations (paths) to the type.
For a function to be well-defined on $\agdacal{K}\;A$ it must respect these
equations (and the type checker enforces this condition).
These extra paths turn the free monoid into the free \emph{commutative}
(\AgdaInductiveConstructor{com}) \emph{idempotent}
(\AgdaInductiveConstructor{dup}) monoid.
The final constructor truncates the type \(\agdacal{K}\;A\) to a (homotopy) set.

The Kuratowski finite subset is a free join semilattice (or, equivalently, a
free commutative idempotent monoid).
More practically, \(\agdacal{K}\) is the abstract data type for finite sets, as
defined in the Boom hierarchy \cite{boomFurtherThoughtsAbstracto1981,
  bunkenburgBoomHierarchy1994}.
However, rather than just being a specification, \(\agdacal{K}\) is fully usable
as a data type in its own right, thanks to HITs.

\begin{definition}[Kuratowski Finiteness]
  A type is Kuratowski finite if there exists a Kuratowski-finite subset of the
  type which contains every element of the type.
  \ExecuteMetaData[agda/Cardinality/Finite/Kuratowski.tex]{kuratowski-finite-def}
\end{definition}

This definition relies on a notion of membership of \(\agdacal{K}\).
That is defined as follows:
\begin{equation*}
  \begin{alignedat}{2}
    x\;\AgdaFunction{\ensuremath{\in}}& \; \AgdaInductiveConstructor{[]}                      &&= \agdabot \\
    x\;\AgdaFunction{\ensuremath{\in}}& \; y \;\AgdaInductiveConstructor{\ensuremath{\dblcolon}}\; \mathit{ys} &&= \AgdaDatatype{\ensuremath{\lVert}}\;(x\;\agdaequiv\;y)\;\AgdaDatatype{\ensuremath{\uplus}}\;x\;\AgdaFunction{\ensuremath{\in}}\;\mathit{ys}\;\AgdaDatatype{\ensuremath{\rVert}}
  \end{alignedat}
\end{equation*}
The \AgdaInductiveConstructor{com} and \AgdaInductiveConstructor{dup}
constructors are handled by proving that the truncated form of \AgdaDatatype{\ensuremath{\uplus}}
itself commutative and idempotent.
The type of propositions is itself a set, satisfying the \AgdaInductiveConstructor{trunc}
constructor.

While Kuratowski finiteness is something of the standard formal definition of
finiteness, it is quite separated from the enumeration-based definitions we have
presented so far.
Nonetheless, it is equivalent to a finiteness definition we have seen already.
\begin{lemma}\label{manifest-enum-kuratowski}
  Kuratowski finiteness is equivalent to truncated manifest enumerability.
  \ExecuteMetaData[agda/Cardinality/Finite/Kuratowski.tex]{manifest-enum-kuratowski}
\end{lemma}
\begin{proof}
  This proof is constructed by providing a pair of functions, to and from each
  side of the equivalence.
  This pair implies an equivalence, because both source and target are
  propositions.
  This proof, as well as its auxiliary lemmas, are also provided in
  \cite{fruminFiniteSetsHomotopy2018}, although there the setting is HoTT
  rather than CuTT.
\end{proof}

Note also that \Cref{manifest-enum-kuratowski} allows us to prove the following:
\ExecuteMetaData[agda/Cardinality/Finite/Kuratowski.tex]{card-iso-kuratowski}
